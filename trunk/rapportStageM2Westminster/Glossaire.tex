\chapter*{Glossaire}
\addcontentsline{toc}{chapter}{Glossaire}
\pagestyle{empty}

\textbf{API} (\textit{Application Programming Interface})\\
Interface qui a pour objet de faciliter le travail d'un programmeur en lui fournissant les outils de base n\'ecessaires \`a tout travail \`a l'aide d'un langage donn\'e.
Elle constitue une interface servant de fondement \`a un travail de programmation plus pouss\'e.

\vspace{0.5cm}

\textbf{Design pattern} (\textit{Patron de conception})\\
Dans un contexte de programmation objet, un \textit{design pattern} d\'ecrit une organisation pratique de classes objets. 
Le but \'etant la r\'eutilisation et la maintenance du code.

\vspace{0.5cm}

\textbf{DOM} (\textit{Document Object Model})\\
API pour les documents HTML ou XML qui fournit une structure arborescente de repr\'esentation du document permettant la modification de son contenu.

\vspace{0.5cm}

\textbf{Framework}\\
Ensemble de fonctions facilitant la cr\'eation de tout ou partie d'un syst\`eme logiciel, ainsi qu'un guide architectural en partitionnant le domaine vis\'e en modules. 
Un \textit{framework} est habituellement impl\'ement\'e \`a l'aide d'un langage \`a objets, bien que cela ne soit pas strictement n\'ecessaire : un \textit{framework} objet fournit ainsi un guide architectural en partitionnant le domaine vis\'e en classes et en d\'efinissant les responsabilit\'es de chacune ainsi que les collaborations entre classes. 

\vspace{0.5cm}

\textbf{IDE} (\textit{Integrated Development Environment})\\
Programme regroupant un ensemble d'outils pour le d\'eveloppement de logiciels.
En r\`egle g\'en\'erale, un IDE regroupe un \'editeur de texte, un compilateur, des outils automatiques de fabrication et souvent un d\'ebogueur.

\vspace{0.5cm}

\textbf{JVM} (\textit{Java Virtual Machine})\\
Environnement d'ex\'ecution ind\'ependant de la plate-forme permettant la conversion d'un bytecode Java (r\'esultat de la compilation d'une classe Java) en langage machine puis son ex\'ecution.

\vspace{0.5cm}

\textbf{LDAP} (\textit{Lightweight Directory Access Protocol})\\
Protocole standard permettant de g\'erer des annuaires. 
Il permet l'acc\`es \`a des bases d'informations sur les utilisateurs, les p\'erif\'eriques et autres composants r\'eseau par l'interm\'ediaire de protocoles TCP/IP.

\vspace{0.5cm}

\textbf{Microsoft Active Directory}\\
Service d'annuaire LDAP, mis au point par Microsoft, pour les syst\`emes d'exploitation Windows.

\vspace{0.5cm}

\textbf{Novell eDirectory}\\
Service d'annuaire LDAP, mis au point par l'entreprise Novell, permettant de g\'erer de fa\c{c}on centralis\'ee l'acc\`es aux ressources des serveurs et ordinateurs au sein d'un m\^eme r\'eseau.

\vspace{0.5cm}

\textbf{Perl}\\
Langage de programmation cr\'e\'e en 1987 reprenant des fonctionnalit\'es du langage C et des langages de \textit{scripts} sed, awk, shell.
C'est un langage interpr\'et\'e adapt\'e dans le traitement et la manipulation de fichiers texte.

\vspace{0.5cm}

\textbf{PHP} (\textit{Personal Home Page} ou \textit{PHP: Hypertext Preprocessor})\\
Langage de \textit{scripts} principalement utilis\'e pour produire des pages Web dynamiques.

\vspace{0.5cm}

\textbf{RSS} (\textit{Rich Site Summary})\\
Un flux RSS est un fichier texte particulier dont le contenu est produit automatiquement en fonction des mises \`a jour d'un site Web.
Les flux RSS sont souvent utilis\'es pour pr\'esenter les titres des derni\`eres informations consultables en ligne dans le cas des sites d'actualit\'e par exemple.
Les flux RSS s'appuit sur le langage XML pour afficher leurs donn\'ees.

\vspace{0.5cm}

\textbf{Servlet}\\
Programme Java qui s'ex\'ecute dynamiquement sur un serveur Web et permet l'extension des fonctions de ce dernier (communication avec un serveur LDAP par exemple).
Les Servlets permettent la gestion de requ\^etes HTTP et de fournir au client une r\'eponse HTTP et ainsi de cr\'eer des pages Web dynamiques.

\vspace{0.5cm}

\textbf{Socket UNIX}\\
Interface de communication de donn\'ees permettant l'\'echange d'informations entre des processus s'ex\'ecutant sur un m\^eme syst\`eme d'exploitation.
Ces sockets ont l'avantage d'\^etre plus rapide que les sockets Internet classiques utilisant un num\'ero de port et accessible depuis le r\'eseau.

\vspace{0.5cm}

\textbf{SQL} (\textit{Structured Query Language})\\
Langage informatique normalis\'e permettant d'effectuer des op\'erations sur des bases de donn\'ees.

\vspace{0.5cm}

\textbf{TCP/IP} (\textit{Transmission Control Protocol / Internet Protocol})\\
Ensemble de protocoles utilis\'es pour le transfert de donn\'ees sur Internet.

\vspace{0.5cm}

\textbf{Wiki}\\
Site Web dont les pages sont modifiables par les visiteurs afin de permettre l'\'ecriture et l'illustration collaboratives des documents num\'eriques qu'il contient.

\vspace{0.5cm}

\textbf{WOL} (\textit{Wake On Lan})\\
Technique permettant de d\'emarrer un ordinateur \'eteint \`a partir d'un r\'eseau. 
Pour un \textit{Wake On Lan}, on parle de r\'eseau local, pour un Wake On Wan, on parle d'Internet.

\vspace{0.5cm}

\textbf{Workflow}\\
Traduction litt\'erale \og{}flux de travail\fg{}, c'est la mod\'elisation et la gestion informatique de l'ensemble des t\^aches \`a accomplir et des diff\'erents acteurs impliqu\'e dans la r\'ealisation d'un processus m\'etier (ou op\'erationnel).

\clearpage
