\documentclass[11pt, a4paper]{report}

\usepackage[frenchb]{babel}
\usepackage[T1]{fontenc}
\usepackage[latin1]{inputenc}
%\usepackage{ucs}
\usepackage{lmodern}
\usepackage{url} % Pour avoir de belles url
\usepackage{geometry}
%\usepackage{slashbox} %backslash dans tableau
\usepackage[table, usenames, dvipsnames]{xcolor} %couleurs
\usepackage{colortbl,hhline}
\usepackage{color}
\usepackage{listings} % Pour mettre du code source
%\usepackage{lscape} %Pour pouvoir passer en paysage
%\usepackage {multicol} %Pour pouvoir faire plusieurs colonnes
%\usepackage{eurosym} %symbole euro
\usepackage{graphicx}
\usepackage{makeidx} %Pour crer un index
%\usepackage{setspace} %Pour l'interligne de 1.5
%\usepackage{shorttoc} %Pour crer un sommaire
%\usepackage{caption}
%\usepackage[font=scriptsize, format=hang]{subcaption}
%\usepackage[babel=true]{csquotes}

%ICI
\usepackage{extensions/picins} %image avec texte a ct
\usepackage[toc,page]{extensions/appendix}
\usepackage{extensions/subfig}
\usepackage{enumitem}

\usepackage{fancyhdr} % Pour les entetes de page
\usepackage[
	pdftex,					% Paramtrage de la navigation
	bookmarks = true,						% Signets
	bookmarksnumbered = true,				% Signets numrots
	pdfpagemode = UseOutlines,						% None, UseThumbs, UseOutlines, Fullscreen
	pdfstartview = Fit,					% FitH, FitV, FitR, FitB, FitBH, FitBV, Fit
	pdfpagelayout = SinglePage,				% SinglePage, OneColumn, TwoColumnLeft, TwoColumnRight
	colorlinks = false,					% Liens en couleur
	urlcolor = black,						% Couleur des liens externes
	pdfborder = {0 0 0}					% Style de bordure : ici, rien
]{hyperref}

\lstset{
	numbers=left,
	numberstyle=\tiny,
%	stepnumber=2,
	numbersep=5pt,
	frame=shadowbox,
	backgroundcolor=\color{grisclair},
	rulesepcolor=\color{gris},
	keywordstyle=\color{darkblue},
	captionpos=b,
	showstringspaces=false,
	numberfirstline=true
}

\lstdefinelanguage{LQL}{
	showspaces=false,
	identifierstyle=\color{black},
	captionpos=b
}

\lstdefinelanguage{JSON}{
	showspaces=false,
	morestring=[b]", %pour les guillemets en bleu
	morecomment=[s]{"?}{?"}, %contenu des guillemets en bleu
	stringstyle=\color{darkblue},
	captionpos=b
}

\lstdefinelanguage{XML}{
	showspaces=false,
	morestring=[b]",
	morestring=[s]{>}{<},
	morecomment=[s]{<?}{?>},
	stringstyle=\color{black},
	identifierstyle=\color{darkblue},
	keywordstyle=\color{cyan},
	morekeywords={id,value,onClick},
	captionpos=b
}

\lstdefinelanguage{plaintext}{
	morekeywords={@}, 
	keywordstyle=\color{darkblue}, 
	captionpos=b
}

\newenvironment{longdescription}
	{\begin{description}[style=unboxed]}
	{\end{description}}

\renewcommand{\labelitemi}{---}
\renewcommand{\labelitemii}{$\star$}

\newcommand\Yuukou{\textit{Yuukou}}
\newcommand\YuukouII{\textit{Yuukou II}}
\newcommand\cad{c'est-\`a-dire}

\renewcommand{\appendixtocname}{Annexes}
\renewcommand{\appendixpagename}{Annexes}

\hypersetup{
	unicode=false,          % non-Latin characters in Acrobat's bookmarks
	pdftoolbar=true,        % show Acrobat's toolbar?
	pdfmenubar=true,        % show Acrobat's menu?
	pdffitwindow=true,     % window fit to page when opened
	pdftitle={Rapport de stage},    % title
	pdfauthor={Meilhac Benoît},     % author
	pdfsubject={Yuukou II},   % subject of the document
	pdfcreator={Meilhac Benoit},   % creator of the document
	pdfproducer={Texlive (pdflatex)}, % producer of the document
	pdfnewwindow=true,      % links in new window
}


\graphicspath{{images/}}
\newcommand{\sommaire}{\shorttoc{Sommaire}{0}}
\makeindex
\definecolor{grisclair}{gray}{0.75}
\definecolor{gris}{gray}{0.5}
\definecolor{orange}{rgb}{1,0.5,0}
\definecolor{vert}{rgb}{0,0.75,0}
\definecolor{darkblue}{rgb}{0.0,0.0,0.6}
\definecolor{cyan}{rgb}{0.0,0.6,0.6}

% Pour les marges de la page
%\geometry{a4paper, top=2.5cm, bottom=3.5cm, left=1.5cm, right=1.5cm, marginparwidth=1.2cm}
\geometry{a4paper}

\parskip=5pt %% distance entre les (paragraphe)
\sloppy %% respecter toujours la marge de droite 

% Pour les pénalités :
\interfootnotelinepenalty=150 %note de bas de page
\widowpenalty=150 %% veuves et orphelines
\clubpenalty=150 

%Pour la longueur de l'indentation des paragraphes
\setlength{\parindent}{15mm}

%%%% debut macro pour enlever le nom chapitre %%%%
\makeatletter
	\def\@makechapterhead#1{%
		\vspace*{50\p@}%
		{\parindent \z@ \raggedright \normalfont
			\interlinepenalty\@M
			\ifnum \c@secnumdepth >\m@ne
			\Huge\bfseries \thechapter\quad
			\fi
			\Huge \bfseries #1\par\nobreak
			\vskip 40\p@
		}}

	\def\@makeschapterhead#1{%
		\vspace*{50\p@}%
		{\parindent \z@ \raggedright
			\normalfont
			\interlinepenalty\@M
			\Huge \bfseries  #1\par\nobreak
			\vskip 40\p@
		}
	}
\makeatother
%%%% fin macro %%%%

\begin{document}
\pagestyle{empty}
\begin{titlepage}
    \centering

    \vspace{5mm}

	\normalsize{
		\textbf{
			D\'epartement informatique\\
			Universit\'e de France-Comt\'e\\
			Stage du 13 f\'evrier au 4 juin\\
			Ann\'ee 2011-2012
		}
	}

    \vspace{5mm}
	
    \Huge{
        \textbf{
            Projet Yuukou II
        }
    }

    \vspace{5mm}

    \normalsize{
        Beno\^it MEILHAC
    }

    \vspace{5mm}
	
    \begin{center} 
        \includegraphics[scale=0.3]{westminsterLogo.jpg}
	\end{center}
	
    \vspace{14mm}

    \Huge{
        \textbf{
            Rapport de stage
        }
	}

	\vspace{20mm}

    \normalsize{
        \textbf{
			School of Electronics and Computer Science\\
            University of Westminster\\
        	115 New Cavendish Street\\
			London W1W 6UW\\
        }
    }

	\vspace{5mm}

	\normalsize{	
	    Tuteur de stage : M. DELAITRE\\
		Responsable de stage : M. HUFFLEN
	}

\end{titlepage}

\clearpage



\pagenumbering{Roman} 
\setcounter{page}{1} 

\chapter*{Remerciements}
\addcontentsline{toc}{chapter}{Remerciements}

Je tiens avant tout \`a remercier mon tuteur de stage M. Thierry DELAITRE pour son accueil \`a l'Universit\'e, pour m'avoir propos\'e un projet aussi int\'eressant, pour avoir mis \`a ma disposition tout ce dont j'avais besoin pour travailler correctement et aussi pour tout le temps qu'il m'a consacr\'e.

Je remercie \'egalement mon responsable de stage M. Jean-Michel HUFFLEN pour m'avoir offert l'opportunit\'e d'effectuer mon stage de fin d'\'etudes \`a Londres mais aussi pour ses conseils concernant la r\'edaction de ce rapport.

J'adresse aussi tous mes remerciements \`a toute l'\'equipe du CPC, avec qui j'ai partag\'e un bureau, pour m'avoir accueilli, pour leur aide et leur bonne humeur malgr\'e la barri\`ere de la langue en d\'ebut de stage.

Merci encore \`a mes camarades francophones Damien HOSTACE et Yacine MAGHEZZI venus me rejoindre en cours de stage pour les bons moments pass\'es ensembles.

J'exprime \'egalement toute ma gratitude \`a tous mes relecteurs pour l'aide qu'ils m'ont apport\'e.

\clearpage


\tableofcontents
\clearpage

\pagestyle{fancy}
\renewcommand{\chaptermark}[1]{\markboth{#1}{}} 
\renewcommand{\sectionmark}[1]{\markright{#1}} 
\pagenumbering{arabic} 
\setcounter{page}{1} 

\chapter{Introduction}

Le cursus professionnel de deuxi\`eme ann\'ee de Master informatique \`a l'Universit\'e de Franche-Comt\'e de Besan\c{c}on offre aux \'etudiant la chance de pouvoir r\'ealiser un stage en entreprise d'une dur\'ee minimum de quatre mois. 
L'universit\'e donne, de plus, l'opportinut\'e d'effectuer ce stage \`a l'\'etranger.

L'anglais est la langue pr\'edominante dans le domaine de l'informatique mais c'est aussi un pr\'e-requis indispensable dans certaines recherches d'emploi.
C'est pour cela que j'ai saisie la chance qu'offrait l'universit\'e pour r\'ealiser mon stage au Royaume-Uni. 
C'est avec l'aide de M. Jean-Michel HUFFLEN, Ma\^itre de conf\'erence \`a l'Universit\'e de Franche-Comt\'e, que j'ai pu int\'egrer pendant quatre mois l'Universit\'e de Westminster \`a Londres o\`u j'ai \'et\'e accueilli par M. Thierry DELAITRE, directeur des infrastructures \`a l'universit\'e, afin de travailler sur le projet \YuukouII.
Ce projet est n\'e d'une collaboration de deux \'equipes de l'universit\'e : l'\'equipe infrastructure de la \textit{School of Electronics and Computer Science} et les services centraux informatique \textit{Information Systems and Library Services}. 
De ce fait, j'ai int\'egr\'e ces deux \'equipes tout en travaillant dans le \textit{Centre for Parallel Computing}.

\Yuukou, venant du japonais et signifiant \textbf{validit\'e, disponibilit\'e, efficacit\'e}, est un projet mis en place dans le but de montrer l'utilisation des salles informatique et d'en garder un historique.
Ainsi, il permet de conna\^itre la disponibilit\'e des salles de l'universit\'e et offre la possibilit\'e de voir les ordinateurs disponibles, occup\'es ou encore \'eteints.

TODO demander confirmation pour ce qui suit

Cependant les principaux d\'eveloppeurs de cette application ne faisant plus parti de l'universit\'e, le projet n'est plus maintenu.
De plus, son principal int\'er\^et serait de faire savoir \`a un \'etudiant les salles qui sont disponibles ainsi que les ordinateurs libres.
Ce qu'il ne fait que tr\`es partiellement actuellement.
Ajout\'e \`a cela que l'universit\'e change le fonctionnement de son infrastructure, ce qui aura, \`a terme, comme cons\'equence de rendre l'application obsol\`ete.

C'est dans ce contexte que s'inscrit le projet \YuukouII. 
Mon stage consiste donc \`a la mise en place d'un service web permettant le suivi des diff\'erents ordinateurs de l'universit\'e et pouvant retourner ces donn\'ees pour un affichage sur un smartphone, ou encore un \'ecran plasma se trouvant \`a l'entr\'ee de chaque b\^atiment par exemple.

Le pr\'esent rapport  le travail r\'ealis\'e au sein de l'universit\'e de Westminster.
Une premi\`ere partie pr\'esentera l'universit\'e et les services que j'ai int\'egr\'e ainsi que ceux que j'ai c\^otoy\'e tout au long du stage.
Une seconde partie tra\^itera du projet \Yuukou{}, avec une pr\'esentation de ses fonctionnalit\'es et de ce \`a quoi il donne acc\`es  ainsi que des raisons qui ont conduit au projet \YuukouII.
Cette partie abordera aussi la d\'ecouverte du sujet de stage.
Une troisi\`eme partie permettra de d\'ecouvrir le sujet avec plus de d\'etails afin de comprendre les choix qui ont \'et\'e r\'ealis\'es dans la quatri\`eme partie dans laquelle les diff\'erentes recherches, le travail accompli et les probl\`emes rencontr\'es seront pr\'esent\'es.
Enfin, une derni\`ere partie donnera un bilan des diff\'erents points retenus lors de ce stage avant de cl\^oturer ce rapport.

\vspace{1.5cm}

\begin{center}
\textit{\underline{NOTE} : Les termes marqu\'es d'une $^*$ sont d\'efinis dans le glossaire.}

\end{center}

\clearpage


\chapter{Lieu de stage}

\section{University of Westminster}

\subsection{Pr\'esentation}

\parpic{
	\begin{minipage}{0.27\textwidth}
		\includegraphics[scale=1.0]{westminsterBlason.jpg}
		\caption{Blason de l'universit\'e}
	\end{minipage}}

L'Universit\'e de Westminster est une universit\'e publique de recherche situ\'ee \`a Londres. 
\`A sa fondation, en 1838, elle portait le nom de \textit{Royal Polytechnic Institution} et \'etait la premi\`ere \'ecole polytechnique \`a ouvrir en Angleterre.
Son but \'etait de fournir une institution o\`u le public peut, \`a moindre co\^ut, acqu\'erir une connaissance pratique des divers arts et branches de la science en rapport avec les fabriquants industriels, les op\'erations mini\`eres et l'\'economie rurale.
Sa fondation est une r\'eaction \`a la popularit\'e grandissante de l'enseignement de type polytechnique en europe continentale. 
Notamment avec l'Allemagne et la \textit{Fachhochschule}, la France et l'\textit{\'Ecole Polytechnique} ou encore les \'Etats-Unis et la \textit{Rensselaer Polytechnic Institute}.

En 1970, la \textit{Royal Polytechnic Institution} devient \textit{Polytechnic of Central London} apr\`es avoir fusionn\'e avec \textit{Holborn College of Law, Languages and Commerce}.

C'est en 1992 que le statut d'universit\'e fut attribu\'e \`a Westminster qui devint \textit{University of Westminster}.

 TODO chiffres sur l'universite comme nombre de diplomes, nombre de ceux qui trouvent rapidement un emploi a la sortie, valorisation du diplome (si le diplome est bien vu dans le pays par exemple

\subsection{Les diff\'erents campus et \'ecoles}
\label{section:campus}
TODO ajouter les 2 satellites LTS et ??
L'universit\'e compte quatre principaux campus, trois dans le centre de Londres : Regent, Cavendish, Marylebone et le quatri\`eme \`a Harrow, \`a l'ouest de Londres.
\begin{itemize}
	\item \textbf{Regent} : situ\'e au 309 Regent Street, c'est le campus le plus ancien de l'universit\'e, il contient deux \'ecoles : 
		\textit{School of Social Sciences, Humanities and Languages} et \textit{School of Law};

	\item \textbf{Cavendish} : situ\'e au 101-115 New Cavendish Street, dans le quartier de Fitzrovia proche de West End de Londres (entre Marylebone, BloomsBury et au nord de Soho), ce campus contient lui aussi trois \'ecoles : 
		\textit{School of Electronics and Computer Science}, \textit{School of Life Sciences} et \textit{Westminster Exchange} dont le but principal est l'am\'elioration de l'\'education;


	\item \textbf{Marylebone} : situ\'e sur Marylebone Road, en face du c\'el\`ebre mus\'ee de cire \textit{Madame Tussauds}, il contient deux \'ecoles :
		\textit{School of Architecture and the Built Environment} et \textit{Westminster Business School};

	\item \textbf{Harrow} : situ\'e dans le village de style victorien Harrow-on-the-Hill surplombant Londres, ce campus contient une \'ecole : 
		\textit{School of Media, Art and Design}.

\end{itemize}

\begin{figure}[!ht]
	\centering
	\subfloat[Fa\c{c}ade]{\includegraphics[scale=0.3]{westminsterRegentExterieur.jpg}}
	\qquad
	\subfloat[Entr\'ee]{\includegraphics[scale=0.242]{westminsterRegentInterieur.jpg}}
	\caption{Campus de Regent Street}

\end{figure}

Outre ces campus, l'universit\'e g\`ere \'egalement le Westminster International University \`a Tachkent en Ouzb\'ekistan ainsi qu'un campus satellite \`a Paris \`a travers l'Acad\'emie Diplomatique de Londres.

\subsection{Quelques informations sur l'universit\'e}

L'universit\'e se situe officiellement au 309 Regent Street London W1B 2HM.
Elle compte en 2011 plus de 20 000 \'etudiants venant de plus de 150 nations diff\'erentes gr\^ace \`a de nombreux programmes d'\'echange avec d'autres universit\'es.

\noindent Parmi ses dipl\^om\'es les plus prestigieux :
\begin{itemize}
	\item Sir \textbf{Alexander Fleming}, biologiste et pharmacologue britannique, prix Nobel de Physiologie ou M\'ed\'ecine en 1945 avec Ernst Boris Chain et Sir Howard Walter Florey pour la d\'ecouverte de la p\'enicilline et de son effet curatif sur diverses maladies infectieuses;
	\item \textbf{Christopher Bailey}, directeur de la cr\'eation chez \textit{Burberry};
	\item \textbf{Charlie Watts}, musicien et batteur des \textit{Rolling Stones}.

\end{itemize}

TODO plus de 300 cours disponibles pour les \'etudiants

\section{School of Electronics and Computer Science}

Comme vu en~\ref{section:campus}, cette \'ecole fait parti du campus Cavendish se trouvant au 101-115 new Cavendish Street. 
La figure~\ref{figure:westminsterNewCavendish} montre une vue ext\'erieure du b\^atiment avec la BT Tower, tour de communication poss\'ed\'ee par l'op\'erateur de t\'el\'ecommunications BT Group, se trouvant juste \`a c\^ot\'e.

\begin{figure}[!ht]
	\centering
	\includegraphics[scale=0.242]{westminsterNewCavendish.jpg}
	\caption{Campus de New Cavendish Street}
	\label{figure:westminsterNewCavendish}

\end{figure}

L'\'ecole propose divers parcours, aussi bien dans le domaine de la recherche que dans celui des \'etudes, allant de l'ing\'enierie \'electronique et informatique, aux math\'ematiques appliqu\'ees.
Ainsi de nombreuses mati\'eres peuvent \^etre abord\'ees dans la formation comme la gestion des syst\'emes d'informations, la programmation parall\`ele et distribu\'ee, l'intelligence artificielle mais aussi le d\'eveloppement de jeux vid\'eos et bien d'autres.

TODO verifier les infos suivantes

Il existe quatre grands p\^oles de recherche au sein de l'\'ecole qui sont :

\textit{Electronic and Communication Engineering}, qui se concentre sur le traitement de signaux ainsi que dans la conception de composants et circuits pour les syst\`emes de communication;

\textit{Operational Research and Intelligent Systems}, dont les activit\'es sont principalement la mod\'elisation quantitative des syst\`emes complexes pour soutenir des processus d\'ecisionnels, la gestion des donn\'ees et des informations, les technologies de base de donn\'ees destin\'ees au processus de gestion et d'interop\'erabilit\'e dans les environnements o\`u les logiciels sont omnipr\'esents;

\textit{Parallel and Distributed Computing}, qui s'int\'eresse \`a la recherche et au d\'eveloppement dans le domaine des calculs parall\`eles et distribu\'es;

\textit{Semantic Computing and Systems Engineering}, regroupe des chercheurs de diff\'erentes disciplines, comme le g\'enie logiciel, les interactions homme-machine, et aborde les aspects th\'eoriques et pratiques de l'informatique s\'emantique.

\section{\'Equipes int\'egr\'ees}
TODO verifier infrastructure team si correct ou non, les deux equipes sont bien separees ? Pas trop compris je pense ce que vous m'avez dit

Le projet \YuukouII{} regroupe deux services de l'universit\'e. 
D'une part l'\textit{Infrastructure Team}, et d'autre part le \textit{Information Systems and Library Services} (ISLS).
Le but de mon stage \'etant de fournir un pilote de ce qu'il est possible de faire afin que mon travail puisse \^etre repris par les deux services et d\'evelopp\'e plus en profondeur.

\subsection{Infrastructure team}

TODO

\subsection{Information Systems and Library Services}

ISLS est un d\'epartement dont le but est de contribuer \`a la qualit\'e de l'\'education dans l'universit\'e de Westminster \`a travers le d\'eveloppement et la livraison de services.

\noindent Le d\'epartement est compos\'e de cinq services qui sont :
\begin{itemize}
	\item Archive Services, qui g\`ere les enregistrements de l'universit\'e;
	\item Corporate Information, qui d\'eveloppe et livre des applications qui prennent en charge le fonctionnement des activit\'es (enregistrement des \'etudiants, emplois du temps, \ldots);
	\item Infrastructure, qui g\`ere l'infrastructure informatique pour fournir des services informatiques et de t\'el\'ecommunication;
	\item Learning and Research Support, qui g\`ere la biblioth\`eque et les services informatiques pour le personnel et les \'etudiants;
	\item Resources and Planning, qui g\`ere les ressources et le planning du d\'epartement.

\end{itemize}

\section{Centre for Parallel Computing}

Durant toute la dur\'ee de mon stage, j'ai travaill\'e dans le \textit{Centre for Parallel Computing} (CPC).
Cette \'equipe appartient au p\^ole recherche \textit{Parallel and Distribued Computing}.

Ce centre se concentre sur la recherche dans la technologie et les applications de calculs parall\`eles et distribu\'ees.
Les activit\'es du cente incluent le d\'eveloppement d'outils et d'environnements pour soutenir le cycle de vie dans le g\'enie logiciel comme la simulation d'\'ev\'enements discrets en parall\`ele.
Les chercheurs travaillent en collaboration avec l'universit\'e hongroise Sztaki.

\noindent Parmi les projets d\'evelopp\'es par le service, on peut citer :
\begin{itemize}
	\item \textbf{Gemlca} (Grid Execution Management for Legacy Code Applications), solution g\'en\'erale dont le but est de deployer le code d'applications existantes, quelque soit le langage, comme un service de grille;
	\item \textbf{NGS} (National Grid Service), solution dont le but est de fournir un acc\`s \'electronique coh\'erent pour les chercheurs du Royaume-Uni \`a toutes les ressources de calculs et de donn\'ees ainsi qu'\`a l'\'equipement n\'ecessaire pour mener \`a bien leurs travaux, ind\'ependamment de l'emplacement de la ressource ou du chercheur;
	\item TODO verifier explications pour SHIWA
	\item \textbf{SHIWA} (SHaring Interoperable Workflows for lages-scale scientific simulations on Available DCIs\protect\footnote{Distributed Computing Infrastructure}), projet qui a pour but l'interop\'erabilit\'e de diff\'erents syst\`emes de workflow$^*$ europ\'een (comme Moteur, P-Grade, Askalon ou Gwes) avec l'aide de l'approche par granularit\'e (grain fin ou grossier).

\end{itemize}

\clearpage


\chapter{Pr\'esentation du sujet}

\section{Le projet \Yuukou}

\begin{figure}[!ht]
	\centering
	\includegraphics[scale=1]{yuukouLogo.jpg}

\end{figure}

\subsection{Pr\'esentation}

Le terme Yuukou comme abord\'e dans l'introduction de ce rapport, vient du japonais \Yuukou{} et signifie validit\'e, disponibilit\'e, efficacit\'e.
C'est un syst\`eme permettant la r\'ecup\'eration des informations depuis les serveurs LDAP\protect\footnote{\textit{Lightweight Directory Access Protocol}}$^*$ de l'Universit\'e afin de comprendre et construire l'infrastructure des ressources et ainsi de voir sa facilit\'e d'utilisation. 
L'architecture du syst\`eme utilise un processus d'apprentissage simple pour d\'eduire et maintenir la structure \`a jour avec un minimum de r\'eglages initiaux.
\Yuukou{} a \'et\'e cr\'e\'e pour montrer l'utilisation des salles informatiques et conserver un historique des informations sur le campus de New Cavendish.

\subsection{Fonctionnement}

Le but principal de \Yuukou{} est d'afficher l'occupation des salles informatiques en se rapprochant autant que possible du comportement d'un syst\`eme fonctionnant en temps r\'eel.
Pour ce faire, l'application a \'et\'e divis\'ee en deux parties  comme le montre la figure~\ref{figure:yuukouFonctionnement}.

\clearpage

\begin{figure}[!ht]
	\centering
	\includegraphics[scale=0.75]{yuukouFonctionnement.jpg}
	\caption{Architecture de \Yuukou}
	\label{figure:yuukouFonctionnement}

\end{figure}

\subsubsection{Premi\`ere partie}

La premi\`ere partie est un programme \'ecrit en Perl$^*$ qui r\'ecup\`ere les informations de connexion des utilisateurs depuis un serveur LDAP$^*$ et qui se charge de construire, modifier ou mettre \`a jour l'architecture r\'eseau de l'Universit\'e tout en stockant les informations dans une base de donn\'ees relationnelle de type MySQL.

Le robot de \Yuukou{} offre deux fonctionnalit\'es : il permet de mettre \`a jour les donn\'ees de connexion \textit{via} le serveur LDAP$^*$ toutes les cinq minutes environ pour une utilisation normale, et de mettre \`a jour le statut des ressources (ordinateurs) toutes les demi-heures, l\`a aussi pour une utilisation normale.

\subsubsection{Deuxi\`eme partie}

La seconde partie est un ensemble de pages Web \'ecrites en PHP\protect\footnote{\textit{Personal Home Page} ou \textit{PHP: Hypertext Preprocessor}}$^*$ et h\'eberg\'ees sur un serveur Web permettant de pr\'esenter les donn\'ees collect\'ees \`a l'utilisateur final.
Ces pages sont de deux types : les pages publiques et les pages priv\'ees.

Les pages publiques sont accessibles par tous les utilisateurs de l'Universit\'e le d\'esirant.
Les pages priv\'ees, quant \`a elles, ne sont accessibles qu'aux administrateurs.

\subsubsection{Les pages publiques}

\noindent Les pages publiques permettent l'affichage des pages suivantes :

\begin{itemize}
	\item une page contenant tous les campus et salles informatiques actuellement utilis\'ees;
	\item une page par campus avec les salles informatiques actuellement utilis\'ees;
	\item une page par campus et d\'epartement avec les salles informatiques actuellement utilis\'ees.

\end{itemize}

\vspace{0.20cm}

Les pages publiques offrent une vision g\'en\'erale de chaque salle : le nombre de ressources totales, disponibles, occup\'ees par un utilisateur et dans un \'etat inconnu.
Il est \`a noter que chacunes des pr\'ec\'edentes pages peut \^etre affich\'ees sur un \'ecran plasma pr\'esent dans les diff\'erents b\^atiments de l'Universit\'e.
La figure~\ref{figure:yuukouPublic} donne un exemple de page publique.

\subsubsection{Les pages priv\'ees}

\noindent Les pages priv\'ees permettent l'affichage des pages suivantes :

\begin{itemize}
	\item une page d'identification \textit{via} LDAP$^*$ pour l'administrateur;
	\item toutes les pages publiques mais b\'en\'eficiant de fonctionnalit\'es suppl\'ementaires :

	\begin{itemize}
		\item liste des ressources \'eteintes;
		\item une fen\^etre permettant d'avoir des informations sur un utilisateur actuellement connect\'e \`a une ressource (identifiant, nom, photo, heure de connexion et dur\'ee de la session);
		\item possibilit\'e d'ajouter des commentaires sur les utilisateurs;
		\item liens vers les statistiques des salles informatiques.

	\end{itemize}

\end{itemize}

\vspace{0.20cm}

La figure~\ref{figure:yuukouAdmin} donne un exemple des fonctionnalit\'es suppl\'ementaires auxquelles un administrateur a acc\`es.

\clearpage

\subsubsection{Vue sur le produit}

\begin{figure}[!ht]
	\centering
	\includegraphics[scale=0.75]{yuukouPublic.jpg}
	\caption{Exemple de page publique de \Yuukou{} rep\'esentant un campus et l'utilisation des salles informatiques d'un d\'epartement}
	\label{figure:yuukouPublic}

\end{figure}

\begin{figure}[!ht]
	\centering
	\includegraphics[scale=0.75]{yuukouAdmin.jpg}
	\caption{Exemple de page priv\'ee de \Yuukou{} montrant en particulier les donn\'ees d'un utilisateur}
	\label{figure:yuukouAdmin}

\end{figure}

\subsection{Quelques chiffres}

Le projet \Yuukou{} permet de surveiller le r\'eseau du campus de New Cavendish soit 43 salles informatiques, ce qui repr\'esente 661 PC. 
Le support des Macintosh n'\'etant pas pris en compte.

\subsection{Changement vers \YuukouII}

L'Universit\'e souhaite reprendre le principe du projet \Yuukou, cependant elle est confront\'ee \`a diff\'erents probl\`emes.
Le premier \'etant que les personnes ayant d\'evelopp\'e le projet ont quitt\'e l'Universit\'e. 
Ce qui signifierait, pour la personne ou l'\'equipe en charge d'\'eventuellement reprendre le projet, de prendre du temps pour se former \`a Perl$^*$, comprendre tout ce qui a \'et\'e d\'ej\`a r\'ealis\'e et ensuite seulement, commencer \`a d\'evelopper.
Actuellement de nombreux projets sont en cours et il n'est pas possible pour une \'equipe de passer du temps sur l'existant.

Un autre probl\`eme est les changements importants, du point de vue infrastructure, qui sont en train d'\^etre mis en place.
En effet, l'Universit\'e qui utilisait eDirectory$^*$ de Novell pour g\'erer ses annuaires LDAP$^*$ a commenc\'e \`a migrer toutes ses donn\'ees vers le syst\`eme Active Directory$^*$ de Microsoft qui est cens\'e \^etre plus efficace. Ce changement rendrait \Yuukou{} obsol\`ete.

Point suivant, la volont\'e de donner acc\`es aux informations sur les salles, pas seulement en utilisant un navigateur Internet, mais aussi et surtout en utilisant un \textit{smartphone} (IPhone ou autres par exemple) ou une tablette (IPad par exemple), chose que l'ancien logiciel ne peut pas fournir.
De ce fait, un \'etudiant aurait \`a tout moment les informations sur les salles \textit{via} son \textit{smartphone} ou sa tablette, si tant est qu'il ait l'un ou l'autre.

\noindent \`A ces pr\'ec\'edents probl\`emes, d'autres viennent s'ajouter :

\begin{itemize}
	\item les Macintosh ne sont pas pris en compte;
	\item le logiciel est assez monolithique donc il deviendrait tr\`es difficile et complexe de vouloir l'\'etendre;
	\item {\Yuukou} ne prend en compte que les donn\'ees en temps r\'eel et ne garde pas un historique;
	\item son utilisation pr\^ete \`a confusion car il n'a aucun interfa\c{c}age avec l'emploi du temps, de ce fait, quand une salle est pr\'esent\'ee comme libre, il n'y a aucun moyen, avec le logiciel, de savoir si un cours s'y d\'eroule ou non.

\end{itemize}

\vspace{0.20cm}

C'est en consid\'erant tous ces points qu'il a \'et\'e d\'ecid\'e d'abandonner le projet \Yuukou{} afin de pouvoir mettre en place \YuukouII{} qui r\'epondrait aux attentes de l'Universit\'e.

\section{Le projet \YuukouII}

{\YuukouII} a pour but de combler les lacunes de {\Yuukou} et d'aller plus loin en termes de fonctionnalit\'es qu'il peut offrir. 
Le projet sera tout d'abord pr\'esent\'e avec les principaux points le composant.
Ensuite seront abord\'ees les diff\'erentes r\'eflexions que les \'equipes int\'eress\'es par le projet ont effectu\'ees.
Ces r\'eflexions donneront lieu aux premi\`eres id\'ees qui ont en \'emerg\'ees.
Enfin, les principales contraintes techniques du projet seront d\'ecrites.

\subsection{Pr\'esentation du projet}

Dans son fonctionnement g\'en\'eral, {\YuukouII} doit permettre de donner \`a un \'etudiant ou toute personne travaillant \`a l'Universit\'e et cherchant \`a utiliser un ordinateur, une vue globale des ressources qui sont disponibles actuellement.
Les donn\'ees devant \^etre bien s\^ur exactes afin que la personne n'ait pas de mauvaise surprise en se rendant dans une salle qu'elle pensait libre.
L'affichage pourra se faire \textit{via} un \textit{smartphone}, une tablette ou encore un navigateur Internet.

\noindent Le but du stage est : 

\begin{itemize}
	\item la conception d'un logiciel permettant la r\'ecup\'eration de donn\'ees concernant les connexions sur les diff\'erentes ressources de l'Universit\'e et cela en temps r\'eel;
	\item la gestion de la persistance de ces donn\'ees;
	\item la cr\'eation d'un maximum de fonctionnalit\'es retournant les informations utiles dans le but d'\^etre exploit\'ees pour l'affichage sur les diff\'erents supports.

\end{itemize} 

\vspace{0.20cm}

La cr\'eation d'applications permettant l'affichage des r\'esultats ne fait pas partie de ce sujet de stage. 
Ici, seule la partie r\'ecup\'eration, stockage et retour des donn\'ees est abord\'ee.

\subsection{R\'eflexions de l'\'equipe technique}

Diff\'erents acteurs de l'Universit\'e int\'eress\'es dans le projet \YuukouII{} ont commenc\'e \`a fixer une liste des fonctionnalit\'es qu'ils aimeraient voir avec l'application finale.

\subsubsection{Concernant les ressources}

\begin{itemize}
	\item Macintosh et Windows;
	\item conna\^itre l'\'etat de la ressource, \'eventuellement la d\'emarrer \`a distance (WOL\protect\footnote{\textit{Wake On Lan}}$^*$);
	\item inclure une surveillance partielle du mat\'eriel et des logiciels;
	\item utilisation des services d'Active Directory$^*$.

\end{itemize}

\subsubsection{Concernant les donn\'ees r\'ecup\'er\'ees}

\begin{itemize}
	\item mettre au point un formalisme avec les autres services de l'Universit\'e concernant les informations sur les salles, campus et d\'epartements;
	\item stocker les informations dans une base de donn\'ees SQL\protect\footnote{\textit{Structured Query Language}}$^*$;
	\item mettre en place des outils pour g\'en\'erer des statistiques \`a partir des donn\'ees stock\'ees (utilisation d'une salle, nombre de connexions par jour dans un mois pour une salle, \ldots).

\end{itemize}

\subsubsection{Concernant les donn\'ees retourn\'ees}

\begin{itemize}
	\item g\'en\'eration de flux RSS\protect\footnote{\textit{Rich Site Summary}}$^*$ pour retourner des informations;
	\item l'affichage doit \^etre en temps r\'eel et doit aussi permettre d'avoir une vue globale dans le temps : utilisation des emplois du temps pour savoir quelle salle est libre et \`a quel moment.

\end{itemize}

\vspace{0.20cm}

Cette liste n'est pas compl\`ete \'etant donn\'e qu'elle ne prend pas en compte la partie \og{}affichage\fg{} des r\'esultats du fait que le projet ne consiste qu'\`a la r\'ecup\'eration, au traitement et au retour de donn\'ees.

L'id\'ee initiale \'etait de d\'evelopper un projet pilote qui permettrait d'avoir une vue sur ce qu'il est possible de faire et sur la fa\c{c}on de le faire.
Il serait ensuite repris par les \'equipes de l'Universit\'e pour \^etre termin\'e.
Ce projet consisterait en la cr\'eation d'un service Web (la notion sera expliqu\'ee plus en d\'etail au \S~\ref{section:serviceWeb}) \'ecrit en C\# et utilisant le \textit{framework}$^*$ .Net de Microsoft.
Le service Web devra offrir un maximum de fonctionnalit\'es et \^etre exploit\'e par une application mobile pour \textit{smartphone} et par un site Web pouvant \^etre projet\'e sur les \'ecrans plasma \`a l'entr\'ee de chaque site.

Cependant, apr\`es r\'eflexion avec M. Thierry DELAITRE, la structure, les objectifs et le projet en g\'en\'eral ont \'et\'e revus.
Dans l'id\'ee initiale, pour conna\^itre l'\'etat d'une ressource, si elle est \'eteinte ou allum\'ee par exemple, il aurait fallu interroger Active Directory$^*$. 
De plus, n'est connu que l'\'etat si un utilisateur est connect\'e.
Il faudrait des traitements suppl\'ementaires pour pouvoir fixer pr\'ecisement l'\'etat d'une ressource, avec un \textit{ping} par exemple pour savoir si la ressource est \'eteinte ou non.

\subsection{Premi\`eres approches}

Une solution simple, rapide et fonctionnelle avait d\'ej\`a \'et\'e mise en place afin de \og{}\textit{monitorer}\fg{}, \cad{} surveiller le fonctionnement des diff\'erentes ressources de certaines salles dans l'Universit\'e.
Nagios est une application permettant d'effectuer une surveillance syst\`eme et r\'eseau.
Il permet de conna\^itre l'\'etat d'une machine ainsi que d'autres informations comme le syst\`eme d'exploitation utilis\'e, la version de Java install\'ee, la charge du processeur, \ldots

En partant de ce logiciel, le projet consiste en la r\'ecup\'eration des donn\'ees de Nagios, leur traitement et le d\'eveloppement des fonctionnalit\'es permettant \`a une application ext\'erieure de pouvoir afficher les informations.

\noindent Les objectifs principaux deviennent les suivants :

\begin{itemize}
	\item cr\'eation d'un service Web en utilisant l'API\protect\footnote{\textit{Application Programming Interface}}$^*$ Java JAX-WS\protect\footnote{\textit{Java API for XML Web Services}};
	\item mise en place d'une m\'ethode de communication avec Nagios;
	\item cr\'eation de la base de donn\'ees permettant l'archivage;
	\item mise en place d'un cycle permettant de traiter les informations r\'ecup\'er\'ees;
	\item d\'efinition de fonctions utiles pour une application cliente;
	\item choix d'une structure de retour des informations pour une application cliente;
	\item faire un lien avec l'emploi du temps des diff\'erentes salles sous surveillance.

\end{itemize}

\subsection{Contraintes techniques}

Deux des principales contraintes du cahier des charges sont l'utilisation de logiciels libres et du langage de d\'eveloppement Java, tout en me laissant un maximum de libert\'e dans les autres choix.

Durant le stage, un ordinateur m'a \'et\'e fourni avec libre choix sur le syst\`eme d'exploitation.
De ce fait j'ai opt\'e pour un Linux Mint 11 nomm\'e \textit{Katya}, que j'ai l'habitude d'utiliser.

Un autre ordinateur, lui contenant un Debian 6.0.5 nomm\'e \textit{Squeeze}, a \'et\'e mis \`a ma disposition en tant que serveur distant h\'ebergeant le service Web ainsi que les diff\'erents outils n\'ecessaires \`a son fonctionnement.
Le serveur distant contient le logiciel Nagios, le serveur Web permettant de faire fonctionner la derni\`ere version stable du service Web ainsi que le syst\`eme de gestion de bases de donn\'ees (SGBD).

Les tests \'etant en premier lieu r\'ealis\'es en local sur la machine de d\'eveloppement et ensuite, quand le fonctionnement \'etait garanti, l'application \'etait d\'eploy\'ee sur le serveur distant.
Des renseignements suppl\'ementaires seront apport\'es au \S~\ref{section:gestionProjet}.


\clearpage

\chapter{D\'ecouverte du projet}

\section{R\'eflexions de l'\'equipe technique}

Diff\'erents acteurs de l'universit\'e int\'eress\'es dans le projet \YuukouII{} ont commenc\'e \`a fixer une liste des fonctionnalit\'es qu'ils aimeraient voir avec l'application finale.

\subsubsection{Concernant les ressources}

\begin{itemize}
	\item Mac et Windows;
	\item conna\^itre l'\`etat de la ressource, \'eventuellement la d\'emarrer \`a distance (WOL\protect\footnote{Wake On Lan}$^*$);
	\item inclure une surveillance partielle du mat\'eriel et des logiciels;
	\item utilisation des services de Active Directory$^*$.

\end{itemize}

\subsubsection{Concernant les donn\'ees r\'ecup\'er\'ees}

\begin{itemize}
	\item mettre au point un formalisme avec les diff\'erents autres services de l'universit\'e concernant les informations sur les salles, campus et d\'epartements;
	\item stocker les informations dans une base de donn\'ees SQL\protect\footnote{Structured Query Language}$^*$;
	\item mettre en place des outils pour g\'en\'erer des statistiques \`a partir des donn\'ees stock\'ees (utilisation d'une salle, nombre de connexion par jour dans un mois pour un salle, \ldots).

\end{itemize}

\subsubsection{Concernant les donn\'ees retourn\'ees}

\begin{itemize}
	\item g\'en\'eration de flux RSS\protect\footnote{Rich Site Summary}$^*$ pour retourner des informations;
	\item l'affichage doit \^etre en temps r\'eel et doit aussi permettre d'avoir un vue globale dans le temps : utilisation des emplois du temps pour savoir quelle salle est libre \`a quel moment.

\end{itemize}

\vspace{0.20cm}

Cette liste n'est pas compl\`ete du fait qu'elle ne prend pas en compte la partie \og{}affichage\fg{} des r\'esultats du fait que le projet ne consiste qu'\`a la r\'ecup\'eration, au traitement et au retour de donn\'ees.

L'id\'ee initiale \'etait de d\'evelopper un projet pilote qui permettrait d'avoir une vue sur ce qu'il est possible de faire et de la fa\c{c}on de le faire.
Il serait ensuite repris par les \'equipes de l'universit\'e pour \^etre termin\'e.
Ce projet consisterait en la cr\'eation d'un service Web$^*$ (la notion sera expliqu\'ee plus en d\'etail en~\ref{section:serviceWeb}) \'ecrit en C\# et utilisant le framework$^*$ .Net de Microsoft.
Le service Web devrait offrir un maximum de fonctionnalit\'es et devra \^etre exploit\'e par une application mobile pour smartphone et par un site web pouvant \^etre projet\'e sur les \'ecrans plasma \`a l'entr\'e de chaque site.

Cependant, apr\`es r\'eflexion avec M. Thierry DELAITRE, la structure, les objectifs et le projet en g\'en\'eral ont \'et\'e revus.
Dans l'id\'ee initiale, pour conna\^itre l'\'etat d'une ressource, il aurait fallu interroger Active Directory$^*$. 
De plus, n'est connu que l'\'etat si un utilisateur est connect\'e.
Il faudrait des traitements suppl\'ementaires pour pouvoir fixer pr\'ecisement l'\'etat d'une ressource, avec un ping par exemple.

\section{Premi\`eres approches}

Une solution simple, rapide et fonctionnelle avait d\'ej\`a \'et\'e mise en place afin de \og{}monitorer\fg{}, \cad{} surveiller le fonctionnement des diff\'erentes ressources de certaines salles dans l'universit\'e.
Nagios est une application permettant d'effetuer une surveillance syst\`eme et r\'eseau.
Il permet de conna\^itre l'\'etat d'une machine ainsi que d'autres informations comme le syst\`eme d'exploitation utilis\'e, la version de Java install\'ee, la charge du processeur, \ldots

En partant de ce logiciel, le projet consiste en la r\'ecup\'eration des donn\'ees de Nagios, leur traitement et le d\'eveloppement des fonctionnalit\'es permettant \`a une application ext\'erieure de pouvoir afficher les informations.

\noindent Les objectifs principaux deviennent les suivants :

\begin{itemize}
	\item cr\'eation d'un service Web en utilisant l'API\protect\footnote{Application Programming Interface}$^*$ Java JAX-WS\protect\footnote{Java API for XML Web Services};
	\item mise en place d'une m\'ethode de communication avec Nagios;
	\item cr\'eation de la base de donn\'ees permettant l'archivage;
	\item mise en place d'un cycle permettant de traiter les informations r\'ecup\'er\'ees;
	\item d\'efinition de fonctions utiles pour une application cliente;
	\item choix d'une structure de retour des informations pour une application cliente;
	\item faire un lien avec l'emploi du temps des diff\'erentes salles sous surveillance;

\end{itemize}

\section{Contraintes techniques}

La volont\'e de M. DELAITRE est d'utiliser des logiciels libre, tout en me laissant un maximum de libert\'e dans les autres choix.

Durant le stage, un ordinateur m'a \'et\'e fourni avec libre choix sur le syst\`eme d'exploitation.
De ce fait j'ai opt\'e pour un Linux Mint 11 nomm\'e \textit{Katya}, que j'ai l'habitude d'utiliser.

Un autre ordinateur, lui contenant un Debian 6.0.5 nomm\'e \textit{Squeeze}, a \'et\'e mis \`a ma disposition en tant que serveur distant h\'ebergeant le service Web ainsi que les diff\'erents les outils n\'ecessaires \`a son fonctionnement.
Le serveur distant contient le logiciel Nagios, le serveur Web permettant de faire fonctionner la derni\`ere version stable du service Web ainsi que le syst\`eme de gestion de base de donn\'ee (SGBD).

Les tests \'etant en premier lieu r\'ealis\'es en local sur la machine de d\'eveloppement et ensuite, quand le fonctionnement \'etait garanti, l'application \'etait d\'eploy\'e sur le serveur distant.
Des renseignements suppl\'ementaires seront apport\'es en~\ref{section:gestionProjet}.



\clearpage


\chapter{La notion de service Web}
\label{section:serviceWeb}

Le projet repose donc sur la cr\'eation d'un service Web.
Son but \'etant de mettre \`a disposition d'un client, des m\'ethodes retournant des informations pour qu'elles soient trait\'ees et affich\'ees.
C'est pourquoi la notion de service Web sera expliqu\'ee dans un premier temps.
Une d\'efinition sera donn\'ee ainsi qu'une architecture basique et la fa\c{c}on dont fonctionne un service Web.
Puis, dans un deuxi\`eme temps, une pr\'esentation de l'API$^*$ utilis\'ee, JAX-WS, pour le d\'eveloppement du service Web sera faite.
Elle permettra de donner un aper\c{c}u des m\'ethodes de d\'eveloppement d'un service Web.

\section{Service Web}

\subsection{D\'efinition}

Un service Web est une application accessible par le r\'eseau, qui permet \`a un client de dialoguer avec le service Web et cela ind\'ependamment de tout langage de programmation et de toute plate-forme d'ex\'ecution.
Un service Web d\'evelopp\'e en Java peut donc \^etre accessible par un client PHP$^*$ par exemple.
C'est un syst\`eme de messagerie standard utilisant le protocole HTTP pour communiquer \`a travers le r\'eseau et \'echangeant des fichiers XML.
Son int\'er\^et resulte dans l'utilisation de normes (HTTP et XML) permettant l'interop\'erabilit\'e des services.
De plus, le fait que les services Web n'imposent pas de mod\`ele de programmation sp\'ecifique permet aux vendeurs d'outils de d\'eveloppement d'offrir des m\'ethodes diff\'erentes et donc de facilement diff\'erencier leurs produits de ceux de leurs concurrents.

\subsection{Architecture}

Lorsqu'on parle de services Web, on parle aussi de SOA, \textit{Service-Oriented Architecture} ou architecture orient\'ee services en fran\c{c}ais.
Une architecture orient\'ee services est un style d'architecture qui a pour objectif une interd\'ependance faible entre diff\'erents agents logiciels (modules, services).
Les services Web poss\`edent trois normes principales qui sont SOAP\protect\footnote{\textit{Simple Object Access Protocol}}, WSDL\protect\footnote{\textit{Web Services Description Language}} et UDDI\protect\footnote{\textit{Universal Description, Discovery and Integration Service}} qui permettent respectivement de conna\^itre la fa\c{c}on dont les messages sont \'echang\'es, leurs descriptions et la fa\c{c}on de les d\'ecouvrir sur le r\'eseau.

\subsubsection{\'Echange de messages}

Dans la grande majorit\'e des services Web, le protocole principal de communication est SOAP.
Il offre le transport d'objets s\'erialis\'es (objets repr\'esent\'es par un flux d'octets) et autres donn\'ees en XML, et l'appel de proc\'edures distantes.

\subsubsection{Description d'un service Web}

Un service Web est d\'ecrit par un document WSDL.
C'est un langage de description bas\'e sur XML qui permet de donner de fa\c{c}on pr\'ecise les d\'etails concernant le service Web :

\begin{itemize}
	\item son protocole de communication utilis\'e;
	\item le format de messages requit pour communiquer avec lui;
	\item les m\'ethodes que le client peut utiliser;
	\item sa localisation.

\end{itemize}

\vspace{0.20cm}

\noindent Un fichier WSDL permet ainsi de savoir comment communiquer avec le service Web.

\subsubsection{D\'ecouverte d'un service Web}

Pour utiliser un service Web, il faut savoir s'il existe.
UDDI est une norme qui d\'efinit le m\'ecanisme pour d\'ecouvrir dynamiquement des services Web.
C'est en fait un annuaire de services permettant de localiser sur le r\'eseau le service Web demand\'e.
\noindent Il comporte :

\begin{itemize}
	\item \textit{des pages blanches}, liste des entreprises ainsi que les informations les concernant;
	\item \textit{des pages jaunes}, liste des services Web de chacunes des entreprises sous le format WSDL
	\item \textit{des pages vertes}, liste des informations techniques pr\'ecises sur les services fournis.

\end{itemize}


\subsection{Fonctionnement}

Le fonctionnement des services Web s'articule autour de trois acteurs principaux illustr\'es par la figure~\ref{figure:schemaServiceWeb}.

\begin{figure}[!ht]
	\centering
	\includegraphics[scale=0.35]{schemaServiceWeb.jpg}
	\caption{Fonctionnement des services Web}
	\label{figure:schemaServiceWeb}

\end{figure}

Les \'etapes impliquant la mise \`a disposition et l'utilisation (consommation) d'un service Web sont les suivantes :

\begin{enumerate}
	\item le service Web d\'ecrit son service en utilisant WSDL. 
	Cette d\'efinition est publi\'ee dans l'annuaire UDDI;
	\item le client envoit une ou plusieurs requ\^etes afin de localiser le service Web \`a l'annuaire et de d\'eterminer comment communiquer avec ce service;
	\item une partie du fichier WSDL fourni par le service Web est pass\'e au client. 
	Le client peut maintenant savoir quelles requ\^etes et r\'eponses envoyer au service Web;
	\item le client utilise le fichier WSDL pour envoyer une requ\^ete au service Web;
	\item le service Web fournit la r\'eponse attendue au client.

\end{enumerate}

\section{L'API JAX-WS}

Java API$^*$ for XML Web Services (JAX-WS) est une API Java permettant de simplifier la cr\'eation, le d\'eveloppement et le d\'eploiement de services Web clients et de services Web prestataires.
Il fonctionne avec un syst\`eme d'annotations sp\'ecifique du code Java.
JAX-WS fait partie de la plate-forme Java EE\protect\footnote{\textit{Java Platform, Entreprise Edition}} de Sun Microsystems.
Anciennement appel\'ee JAX-RPC\protect\footnote{\textit{Java API for XML-based RPC}}, ce changement de nom intervient lors de l'abandon du style RPC\protect\footnote{\textit{Remote Procedure Call}}, des appels de proc\'edures distantes, pour des services Web de style document, avec une transmission de donn\'ees au format XML d\'efinies dans un sch\'ema XML.

Toute la partie communication de JAX-WS est g\'er\'ee avec des messages de type SOAP \`a travers HTTP.
L'API$^*$ permet de cacher toute la complexit\'e des messages, le d\'eveloppeur choisit simplement la mani\`ere d'impl\'ementer le service. 
Pour cela, il existe deux m\'ethodes : \textit{top-down} et \textit{bottom-up}.

\subsubsection{\textit{Top-down}}

La m\'ethode \textit{top-down} permet de d\'evelopper un service Web en partant d'un fichier WSDL.
Ce fichier doit contenir la compl\`ete description de tout le service Web, ensuite le code Java est g\'en\'er\'e automatiquement.
Les classes doivent ensuite \^etre compl\'et\'ees avant d\'eploiement.

\subsubsection{\textit{Bottom-up}}

La m\'ethode \textit{bottom-up} permet, quant \`a elle, de d\'evelopper un service Web en partant du code Java.
Le code Java est d'une part annot\'e, ensuite le fichier WSDL est automatiquement g\'en\'er\'e.
Le service peut enfin \^etre d\'eploy\'e.

\subsubsection{Choix de la m\'ethode et exemple}

C'est la m\'ethode \textit{bottom-up} qui a \'et\'e employ\'ee dans le projet.
Elle permet de ne se soucier que du code Java, le reste \'etant g\'er\'e automatiquement par NetBeans avec la construction du service Web et le d\'eploiement sur le serveur GlassFish.
NetBeans et GlassFish seront d\'ecrits respectivement aux \S~\ref{section:netbeans} et~\ref{section:glassfish}.
Le code~\ref{code:exempleJAXWS} donne un exemple d'annotations sur un code Java.

\clearpage

\begin{figure}[!ht]
	\lstinputlisting[language=Java]{codes/ExempleService.java}
	%\captionof{figure}{Exemple de classe Java annot\'ee avec JAX-WS}
	\caption{Exemple de classe Java annot\'ee avec JAX-WS}
	\label{code:exempleJAXWS}

\end{figure}

JAX-WS \textit{Reference Implementation} est disponible sur son site Internet\cite{biblio:siteJAXWS} en version 2.2.6.

\clearpage


\chapter{Le projet Yuukou 2}

\section{Recherches}

\subsection{Architecture du projet}

\subsection{Nagios}

\subsection{L'IDE NetBeans}

\subsection{Le serveur Web Glassfish}

\subsection{L'API JAX-WS}

\subsection{Le SGBD MySQL}

\subsection{Data Access Object}

\subsection{Le format JSON}

\section{Communication avec Nagios}

\subsection{Le plugin MK Livestatus}

\subsection{Requ\^etes pour Nagios}

\subsection{R\'ecup\'eration des informations}

\section{Le service Web}

\subsection{Mise en place}

\subsection{Le cycle principal}

\subsection{Les fonctionnalit\'es publiques}

\subsection{Les fonctionnalit\'es priv\'ees}

\subsection{Le retour des informations}

\subsection{S\'ecurisation du service Web}

\section{Gestion de l'emploi du temps}

\subsection{Les flux RSS de l'universit\'e}

\subsection{Correspondance des salles}

\section{Gestion de la base de donn\'ees}

\subsection{Structure de la base de donn\'ees}

\subsection{Explications}

\subsection{G\'en\'eration de la configuration Nagios \`a partir de la base de donn\'ees}

\section{Tests du service Web}

\subsection{Mise en place d'un client}

\subsection{Consommation les m\'ethodes}

\section{Am\'eliorations possibles}

\section{Probl\`emes rencontr\'es}


\clearpage


\chapter{Bilan}

\section{Bilan du travail r\'ealis\'e}

\subsection{R\'ecapitulatif du projet}

La figure~\ref{figure:yuukouEtYuukouII} permet de donner une vue d'ensemble sur les principales fonctionnalit\'es dont le projet dispose.
Les cadres pleins repr\'esentent des fonctionnalit\'es dont le principe a \'et\'e repris de {\Yuukou}, les cadres en pointill\'es quant \`a eux, repr\'esentent les nouvelles fonctionnalit\'es qu'apporte \YuukouII.

\begin{figure}[!ht]
	\centering
	\includegraphics[scale=0.375]{yuukouEtYuukouII.jpg}
	\caption{R\'ecapitulatif des fonctionnalit\'es reprises de {\Yuukou} et les nouvelles de \YuukouII}
	\label{figure:yuukouEtYuukouII}

\end{figure}

\subsection{Calendrier du stage}
Arriv\'ee : Lundi 13 F\'evrier
D\'epart : Lundi 4 Juin

semaine 1 : Arriv\'ee et d\'ecouverte du projet, premier entretien avec l'\'equipe technique et leurs attentes
semaine 2 : D\'ecouverte des services Web et choix des outils utilis\'es
semaine 3 : Conception de la base de donn\'ees et d\'ebut d'impl\'ementation du cycle principal de l'application
semaine 4 : fin d'impl\'ementation du cycle principal
semaine 5 : Mise en place de l'emploi du temps + mise en place de SSL
semaine 6 : D\'eveloppement des premi\`eres m\'ethodes Web + fin de mise en place de SSL
semaine 7 : Mise en place du syst\`eme JSON
semaine 8 : 
semaine 9 : 
semaine 10 : D\'ebut de r\'edaction du rapport
semaine 11 : 
semaine 12 : 
semaine 13 : R\'ecup\'eration du catalogue de logiciels de M\'ediaWiki
semaine 14 : Ajout de m\'ethodes Web et am\'elioration du chargement de la base de donn\'ees
semaine 15 : r\'edaction du rapport de stage et maintenance des fonctionnalit\'es du service Web
semaine 16 : fin d'\'ecriture du rapport de stage

\subsection{R\'esultat pour l'Universit\'e}

\subsection{Am\'eliorations possibles}

Le projet {\YuukouII} est loin d'\^etre fini m\^eme si fonctionnel \`a l'heure actuelle.
Il reste diverses fonctionnalit\'es \`a d\'evelopper.
Une de ces fonctionnalit\'es serait de rajouter une s\'ecurit\'e lors de l'acc\`es aux m\'ethodes du service Web par le client.
En effet, il serait int\'eressant que le client s'authentifie.
De ce fait, deux r\^oles pourraient \^etre d\'efinis : administrateur et utilisateur.
Avec cette m\'ethode, un utilisateur ne pourrait jamais se servir d'une m\'ethode dite priv\'ee car actuellement, si un client d\'eveloppe son propre programme pour acc\'eder au service Web {\YuukouII}, une fois le protocole SSL\protect\footnote{\textit{Secure Sockets Layer}} en place, rien ne l'emp\^eche d'utiliser les fonctions qu'il veut.

Des am\'eliorations peuvent aussi \^etre port\'ees sur la fa\c{c}on dont RRDtool a \'et\'e mis en place.
Il ne s'agit que d'un essai pour tester les possibilit\'es, mais il serait int\'eressant de l'int\'egrer pleinement au service Web et surtout au cycle principal d'ex\'ecution.


\section{Bilan humain}

\section{Bilan p\'edagogique}

\clearpage


\chapter{Conclusion}

Ce stage de deuxi\`eme ann\'ee de Master Informatique est une chance de se projeter dans le contexte d'un travail r\'eel en entreprise en travaillant avec des technologies du march\'e.
La libert\'e dont j'ai b\'en\'efici\'e durant toute la dur\'ee du stage m'a permis de d\'ecouvrir de nouvelles technologies et le concept des services Web notamment.
J'ai pu remettre en question certains de mes choix et exercer un avis critique sur mon travail pour au final fournir un service Web fonctionnel.
Celui-ci comprend un cycle principal permettant la r\'ecup\'eration d'informations issues de Nagios pour les stocker dans une base de donn\'ees, ainsi que de multiples fonctions permettant \`a un client d'extraire une partie des donn\'ees qui peuvent lui \^etre utiles afin de pouvoir v\'erifier la disponibilit\'e des salles informatiques dans l'Universit\'e de Westminster.

Ce travail a \'et\'e aussi une chance de travailler en \'equipe avec diff\'erentes personnes.
La communication fut tr\`es importante tout au long du d\'eveloppement afin de pouvoir se mettre d'accord et arriver \`a un r\'esultat tant pour la partie service Web que pour la partie affichage.
Au final, l'application cliente est capable d'afficher toutes les informations concernant la disponibilit\'e des salles en prenant en compte l'emploi du temps.
Elle fournit aussi aux membres des \'equipes techniques des informations suppl\'ementaires comme des graphiques ou des r\'esum\'es sur la situation globale des salles informatiques de l'Universit\'e.

Le stage fut aussi une bonne opportunit\'e d'am\'eliorer mon anglais.
\'Etant dans un environnement en grande partie anglophone, ma compr\'ehension et ma pratique de la langue n'en ont \'et\'e que meilleures.
Ce fut aussi la premi\`ere fois que je venais \`a Londres, j'ai donc pu d\'ecouvrir la ville ainsi que les diff\'erents modes de vie qui la composent.

Je suis au final tr\`es satisfait de ce stage qui m'a apport\'e beaucoup de connaissances ainsi qu'une meilleure pratique de l'anglais.
L'application est disponible actuellement pour les membres de l'Universit\'e.
Dans le cas o\`u des am\'eliorations, des adaptations ou des extensions devaient \^etre effectivement implant\'ees dans le service Web, j'ai tout lieu de penser que la prise en main du programme serait facile, du fait, entre autres, de la pr\'esence de la documentation Java.

\clearpage


%%%%%----------------------------------------
%%%%% Pour la bibliographie
%%%%%----------------------------------------
%%%%% Citer tous les ouvrages/rfrences
\nocite{*}
%%%%% Trier par ordre d'apparition
\bibliographystyle{unsrt}
%%%%% Pour le style de la biblio
%\bibliographystyle{plain}
%%%%% Ecrire la biblio ici
\bibliography{Bibliographie}
\addcontentsline{toc}{chapter}{Bibliographie}

%\printindex

%\appendix

\chapter*{Glossaire}
\addcontentsline{toc}{chapter}{Glossaire}
\pagestyle{empty}

\textbf{API} (\textit{Application Programming Interface})\\
Interface qui a pour objet de faciliter le travail d'un programmeur en lui fournissant les outils de base n\'ecessaires \`a tout travail \`a l'aide d'un langage donn\'e.
Elle constitue une interface servant de fondement \`a un travail de programmation plus pouss\'e.

\vspace{0.5cm}

\textbf{Design pattern} (\textit{Patron de conception})\\
Dans un contexte de programmation objet, un design pattern d\'ecrit une organisation pratique de classes objets. 
Le but \'etant la r\'eutilisation et la maintenance du code.

\vspace{0.5cm}

\textbf{DOM} (\textit{Document Object Model})\\
API pour les documents HTML ou XML qui fournit une structure de repr\'esentation du document permettant la modification de son contenu.

\vspace{0.5cm}

\textbf{Framework}\\
Ensemble de fonctions facilitant la cr\'eation de tout ou d'une partie d'un syst\`eme logiciel, ainsi qu'un guide architectural en partitionnant le domaine vis\'e en modules. 
Un framework est habituellement impl\'ement\'e \`a l'aide d'un langage \`a objets, bien que cela ne soit pas strictement n\'ecessaire : un framework objet fournit ainsi un guide architectural en partitionnant le domaine vis\'e en classes et en d\'efinissant les responsabilit\'es de chacune ainsi que les collaborations entre classes. 

\vspace{0.5cm}

\textbf{IDE} (\textit{Integrated Development Environment})\\
Programme regroupant un ensemble d'outils pour le d\'eveloppement de logiciels.
En r\`egle g\'en\'erale, un IDE regroupe un \'editeur de texte, un compilateur, des outils automatiques de fabrication et souvent un d\'ebogueur.

\vspace{0.5cm}

\textbf{JVM} (\textit{Java Virtual Machine})\\
Environnement d'ex\'ecution ind\'ependant de la plate-forme permettant la conversion d'un bytecode Java (r\'esultat de la compilation d'une classe Java) en langage machine puis son ex\'ecution.

\vspace{0.5cm}

\textbf{LDAP} (\textit{Lightweight Directory Access Protocol})\\
Protocole standard permettant de g\'erer des annuaires. 
Il permet l'acc\`es \`a des bases d'informations sur les utilisateurs, les p\'erif\'eriques et autres composants r\'eseau par l'interm\'ediaire de protocoles TCP/IP.

\vspace{0.5cm}

\textbf{Microsoft Active Directory}\\
Service d'annuaire LDAP, mis au point par Microsoft, pour les syst\`emes d'exploitation Windows.

\vspace{0.5cm}

\textbf{Novell eDirectory}\\
Service d'annuaire LDAP, mis au point par l'entreprise Novell, permettant de g\'erer de fa\c{c}on centralis\'ee l'acc\`es aux ressources des serveurs et ordinateurs au sein d'un m\^eme r\'eseau.

\vspace{0.5cm}

\textbf{Perl}\\
Langage de programmation cr\'e\'e en 1987 reprenant des fonctionnalit\'es du langage C et des langages de scripts sed, awk, shell.
C'est un langage interpr\'et\'e adapt\'e dans le traitement et la manipulation de fichiers texte.

\vspace{0.5cm}

\textbf{PHP} (\textit{Personal Home Page} ou \textit{PHP: Hypertext Preprocessor})\\
Langage de scripts principalement utilis\'e pour produire des pages Web dynamiques.

\vspace{0.5cm}

\textbf{RSS} (\textit{Rich Site Summary})\\
Un flux RSS est un fichier texte particulier dont le contenu est produit automatiquement en fonction des mises \`a jour d'un site Web.
Les flux RSS sont souvent utilis\'es pour pr\'esenter les titres des derni\`eres informations consultables en ligne dans le cas des sites d'actualit\'e par exemple.
Les flux RSS s'appuit sur le langage XML pour afficher leurs donn\'ees.

\vspace{0.5cm}

\textbf{Service Web}\\
Technologie permettant \`a des applications de dialoguer \`a distance via Internet, et ceci ind\'ependamment des plates-formes et des langages sur lesquelles elles reposent.
Pour ce faire, les service Web utilisent un ensemble de protocoles standard d'Internet.

\vspace{0.5cm}

\textbf{Servlet}\\
Programme Java qui s'ex\'ecute dynamiquement sur un serveur Web et permet l'extension des fonctions de ce dernier (communication avec un serveur LDAP par exemple).
Les Servlets permettent la gestion de requ\^etes HTTP et de fournir au client une r\'eponse HTTP et ainsi de cr\'eer des pages Web dynamiques.

\vspace{0.5cm}

\textbf{Socket UNIX}\\
Interface de communication de donn\'ees permettant l'\'echange d'informations entre des processus s'ex\'ecutant sur un m\^eme syst\`eme d'exploitation.
Ces sockets ont l'avantage d'\^etre plus rapide que les sockets Internet classiques utilisant un num\'ero de port et accessible depuis le r\'eseau.

\vspace{0.5cm}

\textbf{SQL} (\textit{Structured Query Language})\\
Langage informatique normalis\'e permettant d'effectuer des op\'erations sur des bases de donn\'ees.

\vspace{0.5cm}

\textbf{TCP/IP} (\textit{Transmission Control Protocol / Internet Protocol})\\
Ensemble de protocoles utilis\'es pour le transfert de donn\'ees sur Internet.

\vspace{0.5cm}

\textbf{WOL} (\textit{Wake On Lan})\\
Technique permettant de d\'emarrer un ordinateur \'eteint \`a partir d'un r\'eseau. 
Pour un Wake On Lan, on parle de r\'eseau local, pour un Wake On Wan, on parle d'Internet.

\vspace{0.5cm}

\textbf{Workflow}\\
Traduction litt\'erale \og{}flux de travail\fg{}, c'est la mod\'elisation et la gestion informatique de l'ensemble des t\^aches \`a accomplir et des diff\'erents acteurs impliqu\'e dans la r\'ealisation d'un processus m\'etier (ou op\'erationnel).

\clearpage


\begin{appendices}

\chapter{Fichiers LQL Nagios}
\label{chapterAnnexe:fichiersLQLNagios}

Cette annexe permet une vue sur les fichiers LQL permettant la r\'ecup\'eration des informations de Nagios. 
Ces informations sont ensuite trait\'ees par le service Web qui se charge de remplir ou actualiser la base de donn\'ees en cons\'equence.

\subsubsection{R\'ecup\'eration des salles}

La requ\^ete~\ref{annexe:nagiosGetHostGroups} permet d'interroger Nagios afin de r\'ecup\'erer les salles qu'il surveille.
Ces salles sont appel\'ees \textit{hostgroups} et contiennent des machines appel\'ees \textit{hosts}.
Parmi ces \textit{hostgroups}, les imprimantes, serveurs et autres ressources sont exclues pour ne retenir que les salles.
La socket Nagios retournera alors le nom de la salle, le nombre de machines qu'elle contient, et la liste des machines qui font partie du groupe.

\vspace{0.20cm}

\begin{figure}[!ht]
	\lstinputlisting[language=LQL]{codes/nagiosGetHostGroups.ngs}
	\caption{Code LQL de r\'ecup\'eration des salles que surveille Nagios}
	\label{annexe:nagiosGetHostGroups}

\end{figure}

\subsubsection{R\'ecup\'eration des machines}

La requ\^ete~\ref{annexe:nagiosGetResources} permet d'interroger Nagios afin de r\'ecup\'erer toutes les machines qui peuvent poss\'eder un utilisateur de connect\'e.
En fait, il est demand\'e la r\'ecup\'eration de tous les services ayant acc\`es \`a l'information \textsf{check\_whoisloggedin}.
Cette information permet de r\'ecup\'erer l'identifiant de la personne connect\'e \`a une machine, si tant est qu'une personne y est connect\'ee.
Il y a pour chaque machine, un service portant ce nom, cela revient donc \`a demander tous les ordinateurs.
La socket Nagios retournera alors le nom de la machine, son adresse IP, le \textsf{host\_groups} donc le nom de la salle \`a laquelle elle appartient, son \'etat et enfin l'utilisateur connect\'e s'il y en a un.

\begin{figure}[!ht]
	\lstinputlisting[language=LQL]{codes/nagiosGetResources.ngs}
	\caption{Code LQL de r\'ecup\'eration des machines que surveille Nagios}
	\label{annexe:nagiosGetResources}

\end{figure}

\subsubsection{R\'ecup\'eration de tous les utilisateurs connect\'es}

La requ\^ete~\ref{annexe:nagiosGetUsersLogged} permet d'interroger Nagios afin de r\'ecup\'erer seulement la liste des utilisateurs connect\'es sur les machines sous surveillance.
Il est demand\'e la r\'ecup\'eration de tous les services parmi lesquels il n'est gard\'e que ceux sur lesquels un utilisateur peut se connecter.
De plus, les messages de Nagios concernant un \'eventuel probl\`eme dans la r\'ecup\'eration de l'information sont \'ecart\'es.
Seul les utilisateurs \og corrects\fg{} sont gard\'es.

\vspace{0.20cm}

\begin{figure}[!ht]
	\lstinputlisting[language=LQL]{codes/nagiosGetUsersLogged.ngs}
	\caption{Code LQL de r\'ecup\'eration des utilisateurs connect\'es sur les machines que surveille Nagios}
	\label{annexe:nagiosGetUsersLogged}

\end{figure}

\chapter{Retours de requ\^ete LQL}
\label{chapterAnnexe:reponseLQLNagios}

Cette annexe permet une vue sur les informations qui sont retourn\'ees par une requ\^ete LQL.
Les informations se pr\'esentent toujours sous la m\^eme forme : une ligne qui est compos\'ee des diff\'erentes colonnes demand\'ees dans la requ\^ete.
Les colonnes sont s\'epar\'ees par un \textsf{";" (point-virgule)}.

\subsubsection{Retour lors d'une r\'ecup\'eration des salles}

La r\'eponse~\ref{annexe:reponseNagiosGetHostGroups} correspond \`a l'ex\'ecution de la requ\^ete~\ref{annexe:nagiosGetHostGroups} sur le serveur contenant Nagios.

\begin{figure}[!ht]
	\centering
	\includegraphics[scale=0.45]{reponseNagiosGetHostGroups.jpg}
	\caption{R\'eponse lors d'une requ\^ete de r\'ecup\'eration de salle}
	\label{annexe:reponseNagiosGetHostGroups}

\end{figure}

\subsubsection{Retour lors d'une r\'ecup\'eration des machines}

La r\'eponse~\ref{annexe:reponseNagiosGetResources} correspond \`a l'ex\'ecution de la requ\^ete~\ref{annexe:nagiosGetResources} sur le serveur contenant Nagios.

\begin{figure}[!ht]
	\centering
	\includegraphics[scale=0.5]{reponseNagiosGetResources.jpg}
	\caption{R\'eponse lors d'une requ\^ete de r\'ecup\'eration des machines}
	\label{annexe:reponseNagiosGetResources}

\end{figure}

\subsubsection{Retour lors d'une r\'ecup\'eration des utilisateurs}

La r\'eponse~\ref{annexe:reponseNagiosGetUsersLogged} correspond \`a l'ex\'ecution de la requ\^ete~\ref{annexe:nagiosGetUsersLogged} sur le serveur contenant Nagios.

\begin{figure}[!ht]
	\centering
	\includegraphics[scale=0.5]{reponseNagiosGetUsersLogged.jpg}
	\caption{R\'eponse lors d'une requ\^ete de r\'ecup\'eration des utilisateurs}
	\label{annexe:reponseNagiosGetUsersLogged}

\end{figure}

\chapter{Base de donn\'ees}
\label{chapterAnnexe:baseDeDonnees}

Cette annexe permet d'avoir une vue sur les relations entre tables ainsi que sur leur contenu.
Dans un premier temps, une description rapide des tables sera faite.
Ensuite, le diagramme de leur agencement les unes avec les autres sera donn\'e en entier.
Enfin, ce diagramme sera d\'ecoup\'e en diff\'erentes parties pour plus de visibilit\'e.

\subsubsection{Description des tables}

\noindent {\YuukouII} est compos\'e de 14 tables :

\begin{itemize}
	\item[\textbf{\textsf{yuukou\_rooms}}] contient les descriptions des salles informatiques;
	\item[\textbf{\textsf{yuukou\_resources}}] contient les descriptions des ordinateurs appartenant \`a une salle particuli\`ere;
	\item[\textbf{\textsf{yuukou\_users}}] contient les descriptions des utilisateurs;
	\item[\textbf{\textsf{yuukou\_last}}] contient les historiques de toutes les connexions pass\'ees sur les ordinateurs;
	\item[\textbf{\textsf{yuukou\_who}}] contient les historiques de toutes les connexions en cours sur les ordinateurs;
	\item[\textbf{\textsf{yuukou\_mapping\_location}}] contient les informations sur les diff\'erents campus de l'Universit\'e, le but \'etant de faire un lien avec la localisation de la table \textsf{yuukou\_rooms} et la description compl\`ete de cette localisation contenue dans la table de \textit{mapping};
	\item[\textbf{\textsf{yuukou\_mapping\_room}}] contient les diff\'erentes correspondances entre le nom des salles telles qu'elles sont appel\'ees dans {\YuukouII} et le nom des salles r\'ecup\'er\'ees avec les emplois du temps;
	\item[\textbf{\textsf{yuukou\_timetables}}] contient les diff\'erentes informations concernant un \'el\'ement de l'emploi du temps;
	\item[\textbf{\textsf{yuukou\_settings}}] contient les diff\'erentes informations de configuration de \YuukouII;
	\item[\textbf{\textsf{yuukou\_groups}}] contient les descriptions des groupes de logiciels;
	\item[\textbf{\textsf{yuukou\_software}}] contient les descriptions des logiciels;
	\item[\textbf{\textsf{yuukou\_groups\_software}}] contient les diff\'erents liens entre un groupe de logiciels et tous les logiciels les composant;
	\item[\textbf{\textsf{yuukou\_rooms\_groups}}] contient les diff\'erents liens entre les salles et tous les groupes de logiciels les composant;
	\item[\textbf{\textsf{yuukou\_roms\_software}}] contient les diff\'erents liens entre les salles et tous les logiciels, n'appartenant pas \`a un groupe de logiciels, les composant.
	
\end{itemize}


\subsubsection{Structure g\'en\'erale}

La figure~\ref{annexe:modeleGeneral} pr\'esente la structure g\'en\'erale du projet \YuukouII.
Elle s'articule en diff\'erentes sous-parties : la partie archivage, logicielle, emploi du temps, mapping et configuration.

\begin{figure}[!ht]
	\centering
	\includegraphics[scale=0.40]{modeleGeneral.png}
	\caption{Structure g\'en\'erale de la base de donn\'ees}
	\label{annexe:modeleGeneral}

\end{figure}

\clearpage

\subsubsection{Partie archivage}

La partie archivage a pour r\^ole le stockage des donn\'ees concernant d'une part l'historique des connexions actuelles et d'autre part, l'historique des connexions pass\'ees.

\begin{figure}[!ht]
	\centering
	\includegraphics[scale=0.35]{modeleArchivage.png}
	\caption{Structure de la partie archivage de la base de donn\'ees}
	\label{annexe:modeleArchivage}

\end{figure}


\subsubsection{Partie logicielle}

La partie logicielle a pour r\^ole de lier les informations de configuration logicielle avec les salles informatiques de l'Universit\'e.

\clearpage

\begin{figure}[!ht]
	\centering
	\includegraphics[scale=0.35]{modeleLogiciel.png}
	\caption{Structure de la partie logicielle de la base de donn\'ees}
	\label{annexe:modeleLogiciel}

\end{figure}


\subsubsection{Partie emploi du temps}

La partie emploi du temps a pour r\^ole de lier les informations d'emploi du temps \`a une salle en passant par une table de mapping faisant le lien entre le nom d'une salle dans {\YuukouII} et le nom d'une salle utilis\'e par les services centraux informatiques lors de la g\'en\'eration des emplois du temps.

\clearpage

\begin{figure}[!ht]
	\centering
	\includegraphics[scale=0.35]{modeleEmploiDuTemps.png}
	\caption{Structure de la partie emploi du temps de la base de donn\'ees}
	\label{annexe:modeleEmploiDuTemps}

\end{figure}

\subsubsection{Partie \textit{mapping} avec les salles}

La partie \textit{mapping} avec les salles a pour r\^ole de faire le lien entre la localisation d'une salle et les informations concernant cette localisation.

\begin{figure}[!ht]
	\centering
	\includegraphics[scale=0.35]{modeleMapping.png}
	\caption{Structure de la partie \textit{mapping} de la base de donn\'ees}
	\label{annexemodeleMapping}

\end{figure}

\subsubsection{Partie configuration}

La partie configuration a pour r\^ole de stocker tous les param\`etres permettant de g\'erer un cycle du service Web pendant son ex\'ecution.

\begin{figure}[!ht]
	\centering
	\includegraphics[scale=0.35]{modeleConfiguration.png}
	\caption{Structure de la partie configuration de la base de donn\'ees}
	\label{annexe:modeleConfiguration}

\end{figure}

\chapter{Exemples de retours JSON}
\label{chapterAnnexe:exempleJSON}

Cette annexe permet d'avoir une vue sur des exemples de retours corrects lors d'appels de fonctions du service Web.
Deux retours de m\'ethodes sont propos\'es ici.

Le code JSON~\ref{annexe:getsitesinformation} correspond au retour de l'appel \`a la fonction \textbf{getSitesInformation ()} du service Web.

\begin{figure}[!ht]
	\centering
	\lstinputlisting[language=JSON]{codes/getSitesInformation.json}
	\caption{Exemple de retour de la fonction \textbf{getSitesInformation ()}}
	\label{annexe:getsitesinformation}

\end{figure}

\clearpage

Le code JSON~\ref{annexe:getlistrooms} correspond au retour de l'appel \`a la fonction \textbf{getListRooms ()} du service Web.

\begin{figure}[!ht]
	\centering
	\lstinputlisting[language=JSON]{codes/getListRooms.json}
	\caption{Exemple de retour de  la fonction \textbf{getListRooms ()}}
	\label{annexe:getlistrooms}

\end{figure}

\end{appendices}

\clearpage


\addcontentsline{toc}{chapter}{Table des figures}
\listoffigures

\clearpage

\addcontentsline{toc}{chapter}{Liste des tableaux}
\listoftables

\clearpage

~
\vfill

\noindent{\LARGE\textbf{R\'esum\'e}}

{\YuukouII} est un projet ayant pour objectif d'afficher la disponibilit\'e des salles informatiques de l'Universit\'e de Westminster \`a Londres.
Il consiste en la cr\'eation d'un service Web permettant de surveiller les ordinateurs de l'Universit\'e \`a l'aide de Nagios et d'archiver toutes les donn\'ees utiles permettant de donner la disponibilit\'e des salles.
Le projet offre diverses fonctionnalit\'es comme une connexion s\'ecuris\'ee, une r\'ecup\'eration des emplois du temps, de la configuration logicielle des salles, des informations concernant les utilisateurs \textit{via} LDAP ou encore la g\'en\'eration de graphes d'utilisation.

Le pr\'esent document rapporte le travail qui a \'et\'e effectu\'e dans le cadre du projet {\YuukouII} au sein de l'Universit\'e, projet r\'ealis\'e pour le stage de fin de cursus de Master 2 \`a l'Universit\'e de Franche-Comt\'e de Besan\c{c}on.

\vspace{0.5cm}

\noindent{\LARGE\textbf{Mots cl\'es}}

Java, service Web, JAX-WS, Nagios, surveillance, LDAP, Yuukou, GlassFish, NetBeans, MySQL.

\vspace{1cm}

\noindent{\LARGE\textbf{Abstract}}

{\YuukouII} is a project which enables to show the availability of computing laboratories within the University of Westminster in London.
During the project, a Web service was implemented to monitor computers of the University with Nagios and which can archive all useful data for showing rooms' availability.
The project provides various functionnalities including secured connection, getting timetables of the University, software configuration of laboratories, information concerning users with LDAP or generation of graphical representation of laboratories utilization.

This document gives a view of all the work done through the project {\YuukouII} within the University during the work placement of the second year of Master degree in the Universit\'e de Franche-Comt\'e of Besan\c{c}on.

\vspace{0.5cm}

\noindent{\LARGE\textbf{Keywords}}

Java, Web service, JAX-WS, Nagios, monitoring, LDAP, Yuukou, GlassFish, NetBeans, MySQL.

\vfill


\end{document}
