\chapter{Pr\'esentation du sujet}

\section{Le projet \textit{Yuukou}}

\begin{figure}[!ht]
	\centering
	\includegraphics[scale=1]{yuukouLogo.jpg}

\end{figure}

\subsection{Pr\'esentation}

Le terme Yuukou vient du japonais \textit{Yuukou} et signifie \textbf{validit\'e, disponibilit\'e, efficacit\'e}. 
C'est un syst\`eme permettant la r\'ecup\'eration des informations depuis les serveurs LDAP de l'universit\'e afin de comprendre et construire l'infrastructure des ressources et ainsi de voir sa facilit\'e d'utilisation. L'architecture du syst\`eme utilise un processus d'apprentissage simple pour d\'eduire et maintenir la structure \`a jour avec un minimum de r\'eglages initiaux.
Yuukou a \'et\'e cr\'e\'e pour montrer l'utilisation des salles informatique et conserver un historique des informations.

\subsection{Fonctionnement}

\subsection{Vue sur le logiciel}

\subsection{Changement vers \textit{Yuukou II}}

current infrastructure

change infrastructure

\section{Le projet \textit{Yuukou II}}

\clearpage
