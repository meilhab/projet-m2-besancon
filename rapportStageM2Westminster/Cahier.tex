\chapter{D\'ecouverte du projet}

\section{R\'eflexions de l'\'equipe technique}

Diff\'erents acteurs de l'universit\'e int\'eress\'es dans le projet \YuukouII{} ont commenc\'e \`a fixer une liste des fonctionnalit\'es qu'ils aimeraient voir avec l'application finale.

\subsubsection{Concernant les ressources}

\begin{itemize}
	\item Mac et Windows;
	\item conna\^itre l'\`etat de la ressource, \'eventuellement la d\'emarrer \`a distance (WOL\protect\footnote{\textit{Wake On Lan}}$^*$);
	\item inclure une surveillance partielle du mat\'eriel et des logiciels;
	\item utilisation des services de Active Directory$^*$.

\end{itemize}

\subsubsection{Concernant les donn\'ees r\'ecup\'er\'ees}

\begin{itemize}
	\item mettre au point un formalisme avec les diff\'erents autres services de l'universit\'e concernant les informations sur les salles, campus et d\'epartements;
	\item stocker les informations dans une base de donn\'ees SQL\protect\footnote{\textit{Structured Query Language}}$^*$;
	\item mettre en place des outils pour g\'en\'erer des statistiques \`a partir des donn\'ees stock\'ees (utilisation d'une salle, nombre de connexion par jour dans un mois pour un salle, \ldots).

\end{itemize}

\subsubsection{Concernant les donn\'ees retourn\'ees}

\begin{itemize}
	\item g\'en\'eration de flux RSS\protect\footnote{\textit{Rich Site Summary}}$^*$ pour retourner des informations;
	\item l'affichage doit \^etre en temps r\'eel et doit aussi permettre d'avoir un vue globale dans le temps : utilisation des emplois du temps pour savoir quelle salle est libre \`a quel moment.

\end{itemize}

\vspace{0.20cm}

Cette liste n'est pas compl\`ete du fait qu'elle ne prend pas en compte la partie \og{}affichage\fg{} des r\'esultats du fait que le projet ne consiste qu'\`a la r\'ecup\'eration, au traitement et au retour de donn\'ees.

L'id\'ee initiale \'etait de d\'evelopper un projet pilote qui permettrait d'avoir une vue sur ce qu'il est possible de faire et de la fa\c{c}on de le faire.
Il serait ensuite repris par les \'equipes de l'universit\'e pour \^etre termin\'e.
Ce projet consisterait en la cr\'eation d'un service Web$^*$ (la notion sera expliqu\'ee plus en d\'etail au \S~\ref{section:serviceWeb}) \'ecrit en C\# et utilisant le framework$^*$ .Net de Microsoft.
Le service Web devrait offrir un maximum de fonctionnalit\'es et devra \^etre exploit\'e par une application mobile pour \textit{smartphone} et par un site web pouvant \^etre projet\'e sur les \'ecrans plasma \`a l'entr\'e de chaque site.

Cependant, apr\`es r\'eflexion avec M. Thierry DELAITRE, la structure, les objectifs et le projet en g\'en\'eral ont \'et\'e revus.
Dans l'id\'ee initiale, pour conna\^itre l'\'etat d'une ressource, il aurait fallu interroger Active Directory$^*$. 
De plus, n'est connu que l'\'etat si un utilisateur est connect\'e.
Il faudrait des traitements suppl\'ementaires pour pouvoir fixer pr\'ecisement l'\'etat d'une ressource, avec un \textit{ping} par exemple.

\section{Premi\`eres approches}

Une solution simple, rapide et fonctionnelle avait d\'ej\`a \'et\'e mise en place afin de \og{}monitorer\fg{}, \cad{} surveiller le fonctionnement des diff\'erentes ressources de certaines salles dans l'universit\'e.
Nagios est une application permettant d'effetuer une surveillance syst\`eme et r\'eseau.
Il permet de conna\^itre l'\'etat d'une machine ainsi que d'autres informations comme le syst\`eme d'exploitation utilis\'e, la version de Java install\'ee, la charge du processeur, \ldots

En partant de ce logiciel, le projet consiste en la r\'ecup\'eration des donn\'ees de Nagios, leur traitement et le d\'eveloppement des fonctionnalit\'es permettant \`a une application ext\'erieure de pouvoir afficher les informations.

\noindent Les objectifs principaux deviennent les suivants :

\begin{itemize}
	\item cr\'eation d'un service Web en utilisant l'API\protect\footnote{\textit{Application Programming Interface}}$^*$ Java JAX-WS\protect\footnote{\textit{Java API for XML Web Services}};
	\item mise en place d'une m\'ethode de communication avec Nagios;
	\item cr\'eation de la base de donn\'ees permettant l'archivage;
	\item mise en place d'un cycle permettant de traiter les informations r\'ecup\'er\'ees;
	\item d\'efinition de fonctions utiles pour une application cliente;
	\item choix d'une structure de retour des informations pour une application cliente;
	\item faire un lien avec l'emploi du temps des diff\'erentes salles sous surveillance;

\end{itemize}

\section{Contraintes techniques}

Une des principales contraintes du cahier des charges est l'utilisation de logiciels libres, tout en me laissant un maximum de libert\'e dans les autres choix.

Durant le stage, un ordinateur m'a \'et\'e fourni avec libre choix sur le syst\`eme d'exploitation.
De ce fait j'ai opt\'e pour un Linux Mint 11 nomm\'e \textit{Katya}, que j'ai l'habitude d'utiliser.

Un autre ordinateur, lui contenant un Debian 6.0.5 nomm\'e \textit{Squeeze}, a \'et\'e mis \`a ma disposition en tant que serveur distant h\'ebergeant le service Web ainsi que les diff\'erents les outils n\'ecessaires \`a son fonctionnement.
Le serveur distant contient le logiciel Nagios, le serveur Web permettant de faire fonctionner la derni\`ere version stable du service Web ainsi que le syst\`eme de gestion de base de donn\'ees (SGBD).

Les tests \'etant en premier lieu r\'ealis\'es en local sur la machine de d\'eveloppement et ensuite, quand le fonctionnement \'etait garanti, l'application \'etait d\'eploy\'ee sur le serveur distant.
Des renseignements suppl\'ementaires seront apport\'es au \S~\ref{section:gestionProjet}.



\clearpage
