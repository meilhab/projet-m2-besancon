\chapter{Conclusion}

Ce stage de deuxi\`eme ann\'ee de Master Informatique est une chance de se projeter dans le contexte d'un travail r\'eel en entreprise en travaillant avec des technologies du march\'e.
La libert\'e dont j'ai b\'en\'efici\'e durant toute la dur\'ee du stage m'a permis de d\'ecouvrir de nouvelles technologies et le concept des services Web notamment.
J'ai pu remettre en question certains de mes choix et exercer un avis critique sur mon travail pour au final fournir un service Web fonctionnel.
Celui-ci comprend un cycle principal permettant la r\'ecup\'eration d'informations issues de Nagios pour les stocker dans une base de donn\'ees, ainsi que de multiples fonctions permettant \`a un client d'extraire une partie des donn\'ees qui peuvent lui \^etre utiles afin de pouvoir v\'erifier la disponibilit\'e des salles informatiques dans l'Universit\'e de Westminster.

Ce travail a \'et\'e aussi une chance de travailler en \'equipe avec diff\'erentes personnes.
La communication fut tr\`es importante tout au long du d\'eveloppement afin de pouvoir se mettre d'accord et arriver \`a un r\'esultat tant pour la partie service Web que pour la partie affichage.
Au final, l'application cliente est capable d'afficher toutes les informations concernant la disponibilit\'e des salles en prenant en compte l'emploi du temps.
Elle fournit aussi aux membres des \'equipes techniques des informations suppl\'ementaires comme des graphiques ou des r\'esum\'es sur la situation globale des salles informatiques de l'Universit\'e.

Le stage fut aussi une bonne opportunit\'e d'am\'eliorer mon anglais.
\'Etant dans un environnement en grande partie anglophone, ma compr\'ehension et ma pratique de la langue n'en ont \'et\'e que meilleures.
Ce fut aussi la premi\`ere fois que je venais \`a Londres, j'ai donc pu d\'ecouvrir la ville ainsi que les diff\'erents modes de vie qui la composent.

Je suis au final tr\`es satisfait de ce stage qui m'a apport\'e beaucoup de connaissances ainsi qu'une meilleure pratique de l'anglais.
L'application est disponible actuellement pour les membres de l'Universit\'e.
Dans le cas o\`u des am\'eliorations, des adaptations ou des extensions devaient \^etre effectivement implant\'ees dans le service Web, j'ai tout lieu de penser que la prise en main du programme serait facile, du fait, entre autres, de la pr\'esence de la documentation Java.

\clearpage
