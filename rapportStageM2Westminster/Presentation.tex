\chapter{Lieu de stage}

\section{University of Westminster}

\subsection{Pr\'esentation}

L'Universit\'e de Westminster est une universit\'e publique de recherche situ\'ee \`a Londres. 
\`A sa fondation, en 1838, elle portait le nom de \textit{Royal Polytechnic Institution} et \'etait la premi\`ere \'ecole polytechnique \`a ouvrir en Angleterre.
Son but \'etait de fournir une institution o\`u le public peut, \`a moindre co\^ut, acqu\'erir une connaissance pratique des divers arts et branches de la science en rapport avec les fabriquants, les op\'erations mini\`eres et l'\'economie rurale.
Sa fondation est une r\'eaction \`a la popularit\'e grandissante de l'enseignement de type polytechnique en europe continentale. 
Notamment avec l'Allemagne et la \textit{Fachhochschule}, la France et l'\textit{\'Ecole Polytechnique} ou encore les \'Etats-Unis et la \textit{Rensselaer Polytechnic Institute}.

En 1970, la \textit{Royal Polytechnic Institution} devient \textit{Polytechnic of Central London} apr\`es avoir fusionn\'e avec \textit{Holborn College of Law, Languages and Commerce}.

C'est en 1992 que le statut d'universit\'e est attribu\'e \`a Westminster qui devient \textit{University of Westminster}.

% TODO chiffres sur l'universite comme nombre d'etudiants, nombre de diplomes, nombre de ceux qui trouvent rapidement un emploi a la sortie, \ldots

\subsection{Quelques informations sur l'universit\'e}

L'universit\'e se situe officiellement au 309 Regent Street London W1B 2HM.
Elle compte en 2011 plus de 20000 \'etudiants venant de plus de 150 nations diff\'erentes gr\^ace \`a de nombreux programmes d'\'echange avec d'autres universit\'es.

\noindent Parmi ses dipl\^om\'es les plus prestigieux :
\begin{itemize}
	\item Sir \textbf{Alexander Fleming}, biologiste et pharmacologue britannique, prix Nobel de Physiologie ou M\'ed\'ecine en 1945 avec Ernst Boris Chain et Sir Howard Walter Florey pour la d\'ecouverte de la p\'enicilline et de son effet curatif sur diverses maladies infectieuses;
	\item \textbf{Christopher Bailey}, directeur de la cr\'eation chez \textit{Burberry};
	\item \textbf{Charlie Watts}, musicien et batteur des \textit{Rolling Stones}.

\end{itemize}

\subsection{Les diff\'erents campus et \'ecoles}

L'universit\'e compte quatre principaux campus, trois dans le centre de Londres : Regent, Cavendish, Marylebone et le quatri\`eme \`a Harrow, \`a l'ouest de Londres.

\begin{itemize}
	\item \textbf{Regent} : situ\'e au 309 Regent Street, c'est le campus le plus ancien de l'universit\'e, il contient deux \'ecoles : 
	\begin{itemize}
		\item School of Social Sciences, Humanities and Languages;
		\item School of Law.

	\end{itemize}

	\item \textbf{Cavendish} : situ\'e au 101-115 New Cavendish Street, dans le quartier de Fitzrovia proche de West End de Londres (entre Marylebone, BloomsBury et au nord de Soho), ce campus contient lui aussi deux \'ecoles :
	\begin{itemize}
		\item School of Electronics and Computer Science;
		\item School of Life Sciences.

	\end{itemize}

	\item \textbf{Marylebone} : situ\'e sur Marylebone Road, en face du c\'el\`ebre mus\'ee de cire \textit{Madame Tussauds}, il contient deux \'ecoles ;
	\begin{itemize}
		\item School of Architecture and the Built Environment;
		\item Westminster Business School.

	\end{itemize}

	\item \textbf{Harrow} : situ\'e dans le village de style victorien Harrow-on-the-Hill surplombant Londres, ce campus contient une \'ecole : School of Media, Art and Design.

\end{itemize}

Outre ces campus, l'universit\'e g\`ere \'egalement le Westminster International University \`a Tachkent en Ouzb\'ekistan ainsi qu'un campus satellite \`a Paris \`a travers l'Acad\'emie Diplomatique de Londres.

\section{School of Electronics and Computer Science}

% TODO peut etre une partie sur l'equipe integree ? si on considere que je fais parti d'une equipe

\clearpage
