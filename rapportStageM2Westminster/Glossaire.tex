\chapter*{Glossaire}
\addcontentsline{toc}{chapter}{Glossaire}

\textbf{LDAP} (\textit{Lightweight Directory Access Protocol})\\
Protocole standard permettant de g\'erer des annuaires. 
Il permet l'acc\`es \`a des bases d'informations sur les utilisateurs, les p\'erif\'eriques et autres composants r\'eseau par l'interm\'ediaire de protocoles TCP/IP.

\vspace{0.5cm}

\textbf{TCP/IP} (\textit{Transmission Control Protocol / Internet Protocol})\\
Ensemble de protocoles utilis\'es pour le transfert de donn\'ees sur Internet.

\vspace{0.5cm}

\textbf{Novell eDirectory}\\
Service d'annuaire LDAP, mis au point par l'entreprise Novell, permettant de g\'erer de fa\c{c}on centralis\'ee l'acc\`es aux ressources des serveurs et ordinateurs au sein d'un m\^eme r\'eseau.

\vspace{0.5cm}

\textbf{Microsoft Active Directory}\\
Service d'annuaire LDAP, mis au point par Microsoft, pour les syst\`emes d'exploitation Windows.

\vspace{0.5cm}

\textbf{IDE} (\textit{Integrated Development Environment})\\
Programme regroupant un ensemble d'outils pour le d\'eveloppement de logiciels.
En r\`egle g\'en\'erale, un IDE regroupe un \'editeur de texte, un compilateur, des outils automatiques de fabrication et souvent un d\'ebogueur.

\vspace{0.5cm}

\textbf{Perl}\\
Langage de programmation cr\'e\'e en 1987 reprenant des fonctionnalit\'es du langage C et des langages de scripts sed, awk, shell.
C'est un langage interpr\'et\'e adapt\'e dans le traitement et la manipulation de fichiers texte.

\vspace{0.5cm}

\textbf{PHP} (\textit{Personal Home Page} ou \textit{PHP: Hypertext Preprocessor})\\
Langage de scripts principalement utilis\'e pour produire des pages Web dynamiques.

\clearpage
