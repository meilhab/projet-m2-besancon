\chapter*{Glossaire}
\addcontentsline{toc}{chapter}{Glossaire}

\textbf{API} (\textit{Application Programming Interface})\\
Interface qui a pour objet de faciliter le travail d'un programmeur en lui fournissant les outils de base n\'ecessaires \`a tout travail \`a l'aide d'un langage donn\'e.
Elle constitue une interface servant de fondement \`a un travail de programmation plus pouss\'e.

\vspace{0.5cm}

\textbf{Framework}\\
Ensemble de fonctions facilitant la cr\'eation de tout ou d'une partie d'un syst\`eme logiciel, ainsi qu'un guide architectural en partitionnant le domaine vis\'e en modules. 
Un framework est habituellement impl\'ement\'e \`a l'aide d'un langage \`a objets, bien que cela ne soit pas strictement n\'ecessaire : un framework objet fournit ainsi un guide architectural en partitionnant le domaine vis\'e en classes et en d\'efinissant les responsabilit\'es de chacune ainsi que les collaborations entre classes. 

\vspace{0.5cm}

\textbf{IDE} (\textit{Integrated Development Environment})\\
Programme regroupant un ensemble d'outils pour le d\'eveloppement de logiciels.
En r\`egle g\'en\'erale, un IDE regroupe un \'editeur de texte, un compilateur, des outils automatiques de fabrication et souvent un d\'ebogueur.

\vspace{0.5cm}

\textbf{LDAP} (\textit{Lightweight Directory Access Protocol})\\
Protocole standard permettant de g\'erer des annuaires. 
Il permet l'acc\`es \`a des bases d'informations sur les utilisateurs, les p\'erif\'eriques et autres composants r\'eseau par l'interm\'ediaire de protocoles TCP/IP.

\vspace{0.5cm}

\textbf{Microsoft Active Directory}\\
Service d'annuaire LDAP, mis au point par Microsoft, pour les syst\`emes d'exploitation Windows.

\vspace{0.5cm}

\textbf{Novell eDirectory}\\
Service d'annuaire LDAP, mis au point par l'entreprise Novell, permettant de g\'erer de fa\c{c}on centralis\'ee l'acc\`es aux ressources des serveurs et ordinateurs au sein d'un m\^eme r\'eseau.

\vspace{0.5cm}

\textbf{Perl}\\
Langage de programmation cr\'e\'e en 1987 reprenant des fonctionnalit\'es du langage C et des langages de scripts sed, awk, shell.
C'est un langage interpr\'et\'e adapt\'e dans le traitement et la manipulation de fichiers texte.

\vspace{0.5cm}

\textbf{PHP} (\textit{Personal Home Page} ou \textit{PHP: Hypertext Preprocessor})\\
Langage de scripts principalement utilis\'e pour produire des pages Web dynamiques.

\vspace{0.5cm}

\textbf{RSS} (\textit{Rich Site Summary})\\
Un flux RSS est un fichier texte particulier dont le contenu est produit automatiquement en fonction des mises \`a jour d'un site Web.
Les flux RSS sont souvent utilis\'es pour pr\'esenter les titres des derni\`eres informations consultables en ligne dans le cas des sites d'actualit\'e par exemple.
Les flux RSS s'appuit sur le langage XML pour afficher leurs donn\'ees.

\vspace{0.5cm}

\textbf{Service Web}\\
Technologie permettant \`a des applications de dialoguer \`a distance via Internet, et ceci ind\'ependamment des plates-formes et des langages sur lesquelles elles reposent.
Pour ce faire, les service Web utilisent un ensemble de protocoles standard d'Internet.

\vspace{0.5cm}

\textbf{Servlet}\\
Programme Java qui s'ex\'ecute dynamiquement sur un serveur Web et permet l'extension des fonctions de ce dernier (communication avec un serveur LDAP par exemple).
Les Servlets permettent la gestion de requ\^etes HTTP et de fournir au client une r\'eponse HTTP et ainsi de cr\'eer des pages Web dynamiques.


\vspace{0.5cm}

\textbf{SQL} (\textit{Structured Query Language})\\
Langage informatique normalis\'e permettant d'effectuer des op\'erations sur des bases de donn\'ees.

\vspace{0.5cm}

\textbf{TCP/IP} (\textit{Transmission Control Protocol / Internet Protocol})\\
Ensemble de protocoles utilis\'es pour le transfert de donn\'ees sur Internet.

\vspace{0.5cm}

\textbf{WOL} (\textit{Wake On Lan})\\
Technique permettant de d\'emarrer un ordinateur \'eteint \`a partir d'un r\'eseau. 
Pour un Wake On Lan, on parle de r\'eseau local, pour un Wake On Wan, on parle d'Internet.

\vspace{0.5cm}

\textbf{Workflow}\\
Traduction litt\'erale \og{}flux de travail\fg{}, c'est la mod\'elisation et la gestion informatique de l'ensemble des t\^aches \`a accomplir et des diff\'erents acteurs impliqu\'e dans la r\'ealisation d'un processus m\'etier (ou op\'erationnel).

\vspace{0.5cm}

\textbf{XML} (\textit{eXtensible Markup Language})\\
Langage de balisage g\'en\'erique permettant de mettre en forme des documents.



\clearpage
