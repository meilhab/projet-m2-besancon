\chapter{Introduction}

Le cursus professionnel de deuxi\`eme ann\'ee du Master informatique \`a l'Universit\'e de Franche-Comt\'e de Besan\c{c}on inclut la r\'ealisation d'un stage en entreprise d'une dur\'ee minimum de quatre mois. 
L'Universit\'e donne, de plus, l'opportunit\'e d'effectuer ce stage \`a l'\'etranger.

L'anglais est la langue pr\'edominante dans le domaine de l'informatique mais c'est aussi un pr\'e-requis indispensable dans certaines recherches d'emploi.
C'est pour cela que j'ai saisi la chance qu'offrait l'Universit\'e pour r\'ealiser mon stage au Royaume-Uni. 
C'est avec l'aide de M. Jean-Michel HUFFLEN, Ma\^itre de conf\'erences \`a l'Universit\'e de Franche-Comt\'e, que j'ai pu int\'egrer pendant quatre mois l'Universit\'e de Westminster \`a Londres o\`u j'ai \'et\'e accueilli par M. Thierry DELAITRE, directeur des infrastructures \`a l'Universit\'e, afin de travailler sur le projet \YuukouII.

Ce projet est n\'e d'une collaboration de deux \'equipes de l'Universit\'e : l'\textit{Infrastructure Team} de la \textit{School of Electronics and Computer Science} et les services centraux informatique \textit{Information Systems and Library Services}. 
De ce fait, j'ai int\'egr\'e ces deux \'equipes tout en travaillant dans le \textit{Centre for Parallel Computing}.

\Yuukou, venant du japonais et signifiant validit\'e, disponibilit\'e, efficacit\'e, est un projet mis en place dans le but de montrer l'utilisation des salles informatiques et d'en garder un historique.
Ainsi, il permet de conna\^itre la disponibilit\'e des salles de l'Universit\'e et offre la possibilit\'e de voir les ordinateurs disponibles, occup\'es ou encore \'eteints.

Cependant les principaux d\'eveloppeurs de cette application ne faisant plus partie de l'Universit\'e, le projet n'est plus maintenu.
De plus, son principal int\'er\^et serait de faire savoir \`a un \'etudiant les salles qui sont disponibles ainsi que les ordinateurs libres.
Ce qu'il ne fait que tr\`es partiellement actuellement.
En outre, l'Universit\'e change le fonctionnement de son infrastructure, ce qui aura, entre autres cons\'equences de rendre l'application obsol\`ete.

C'est dans ce contexte que s'inscrit le projet \YuukouII. 
Mon stage consiste donc \`a la mise en place d'un service Web permettant le suivi des diff\'erents ordinateurs de l'Universit\'e et pouvant retourner ces donn\'ees pour un affichage sur un \textit{smartphone}, ou encore un \'ecran plasma se trouvant \`a l'entr\'ee de chaque b\^atiment par exemple.

Le pr\'esent rapport concerne le travail r\'ealis\'e au sein de l'Universit\'e de Westminster.
Une premi\`ere partie pr\'esentera l'Universit\'e et les services que j'ai int\'egr\'es ainsi que ceux que j'ai c\^otoy\'es tout au long du stage.
Une seconde partie tra\^itera du projet \Yuukou{}, avec une pr\'esentation de ses fonctionnalit\'es et de ce \`a quoi il donne acc\`es  ainsi que des raisons qui ont conduites au projet \YuukouII.
Cette partie abordera aussi la d\'ecouverte du sujet de stage.
Une troisi\`eme partie permettra de d\'ecouvrir le sujet avec plus de d\'etails afin de comprendre les choix qui ont \'et\'e r\'ealis\'es dans les autres parties du rapport.
La quatri\`eme partie mettra en avant la notion de service Web, le point central du projet.
Ensuite, une cinqui\`eme partie pr\'esentera les diff\'erentes recherches qui ont \'et\'e faites, le travail accompli et les probl\`emes rencontr\'es.
Enfin, une derni\`ere partie donnera un bilan des diff\'erents points retenus lors de ce stage avant de cl\^oture ce rapport.

\vspace{1.5cm}

\begin{center}
\textit{\underline{NOTE} : Les termes marqu\'es d'une ast\'erisque($^*$) sont d\'efinis dans le glossaire.}

\end{center}

\clearpage
