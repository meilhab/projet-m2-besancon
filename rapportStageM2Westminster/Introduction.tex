\chapter{Introduction}

Le cursus professionnel de deuxi\`eme ann\'ee de Master informatique \`a l'Universit\'e de Franche-Comt\'e de Besan\c{c}on offre aux \'etudiant la chance de pouvoir r\'ealiser un stage en entreprise d'une dur\'ee minimum de quatre mois. 
L'universit\'e donne, de plus, l'opportinut\'e d'effectuer ce stage \`a l'\'etranger.

L'anglais est la langue pr\'edominante dans le domaine de l'informatique mais c'est aussi un pr\'e-requis indispensable dans certaines recherches d'emploi.
C'est pour cela que j'ai saisie la chance qu'offrait l'universit\'e pour r\'ealiser mon stage au Royaume-Uni. 
C'est avec l'aide de M. Jean-Michel HUFFLEN, Ma\^itre de conf\'erence \`a l'Universit\'e de Franche-Comt\'e, que j'ai pu int\'egrer pendant quatre mois l'Universit\'e de Westminster \`a Londres o\`u j'ai \'et\'e accueilli par M. Thierry DELAITRE, directeur des infrastructures \`a l'universit\'e, afin de travailler sur le projet \YuukouII.

TODO parler de l'equipe ?
parti de 2 equipes : infrastructure team de la faculte ECS et colloboration avec les services centraux informatique ISLS
fais stage avec personne du CPC

\Yuukou, venant du japonais et signifiant \textbf{validit\'e, disponibilit\'e, efficacit\'e}, est un projet mis en place dans le but de montrer l'utilisation des salles informatique et d'en garder un historique.
Ainsi, il permet de conna\^itre la disponibilit\'e des salles de l'universit\'e et offre la possibilit\'e de voir les ordinateurs disponibles, occup\'es ou encore \'eteints.

TODO demander confirmation pour ce qui suit

Cependant les principaux d\'eveloppeurs de cette application ne faisant plus parti de l'universit\'e, le projet n'est plus maintenu.
De plus, son principal int\'er\^et serait de faire savoir \`a un \'etudiant les salles qui sont disponibles ainsi que les ordinateurs libres.
Ce qu'il ne fait que tr\`es partiellement actuellement.
Ajout\'e \`a cela que l'universit\'e change le fonctionnement de son infrastructure, ce qui aura, \`a terme, comme cons\'equence de rendre l'application obsol\`ete.

C'est dans ce contexte que s'inscrit le projet \YuukouII. 
Mon stage consiste donc \`a la mise en place d'un service web permettant le suivi des diff\'erents ordinateurs de l'universit\'e et pouvant retourner ces donn\'ees pour un affichage sur un smartphone, ou encore un \'ecran plasma se trouvant \`a l'entr\'ee de chaque b\^atiment par exemple.

 TODO completer avec l'annonce du plan

\vspace{1.5cm}

\noindent Les termes marqu\'e d'une $^*$ sont d\'efinis dans le glossaire se trouvant \`a la fin de ce rapport.

\clearpage
